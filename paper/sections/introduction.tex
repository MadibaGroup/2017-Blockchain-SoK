% !TEX root = ../main.tex

\section{Introduction}

In 2008, an author using the pseudonym Satoshi Nakamoto wrote a white paper describing Bitcoin, a new decentralized cryptocurrency~\cite{Nak08}. Unlike past attempts at forming a cryptocurrency--- attempts which relied on pre-established trusted entities for the system to operate correctly --- Bitcoin's design runs on the open internet with no one in charge, while maintaining uptight security. While the building blocks of Bitcoin were not novel, the composition of these properties into a single system was a meaningful contribution~\cite{Narayanan17} and Bitcoin became the first cryptocurrency to achieve widespread attention.

In response to Bitcoin's success, the technology was quickly dissected to understand how it works and what is new. Its most innovative component has been labeled ``the blockchain,'' a decentralized approach for participants to agree upon and lock in data and computation.

% Cite Bruce Schneier wired article
When we read technology news, we are most commonly left with the cheery impression that blockchain technology reduces or even completely eliminates trust. The use cases of such an innovation stretch the imagination! Occasionally, we get lucky and hear a contrarian take~\cite{Sch19}. The truth about trust is that it's complicated. Blockchain does eliminate specific, narrow reliances on trust, but it might shift it into new and different assumptions---assumptions that might be better or worst for each specific use case. Likewise, there are not a lot of single-sentence talking points that will be accurate about blockchain technology's efficiency, security, cost, etc. 

It is clear that we need a more nuanced discussion on this technology. We frequently hear business executives, government leaders, investors, and researchers ask the following three questions: (1) what exactly is blockchain technology, (2) what capabilities does it provide, and (3) what are good applications? Our goal for this article is to thoroughly answer these questions,  provide a holistic overview of blockchain technology that separates hype from reality, and propose a useful lexicon for discussing the specifics of blockchain technology in the future.

%Concretely, our key contributions are:

%\begin{enumerate}
%	\item \textbf{Providing a holistic overview of Blockchain technology's technical properties and the capabilities they enable.}
%	We organize the properties of Blockchain technology into three groups: shared governance and operation, verifiable state, and resilience to data loss.
%	Taken together, these properties differentiate Blockchain technology from other distributed technologies.
%	Using these properties, systems built with Blockchain technology have easy access to a range of important capabilities: full-system provenance, auditability, access control, psuedonymity, automatic execution (i.e., smart contracts), and data discoverability.
	
%	Blockchain systems are governed and operated by a distributed and decentralized collective of entities that ensure that all operations are valid.
%	These operations are recorded in an append-only ledger that provides full-system provenance.
%	This ledger is also replicated by a set of governing parties, providing resilience against data loss and malicious modification of data.
%	We also discuss associated technical properties that Blockchain makes use of.
	
%	\item \textbf{Identifying the capabilities provided by Blockchain technology.}
% 	Blockchain provides four important capabilities.
%	Blockchain systems inherently track full-system provenance, and this allows the system to be audited by miners and potentially by other parties.
%	Provenance is also useful in tracking assets monitored or managed by the Blockchain system (e.g., supply chain management, cryptocurrencies).
%	Blockchain technology provides access control for the assets it monitors or manages and can allow operations on these assets to be pseudonymous.
%	Provenance information and replication of that information among all miners provides a third important capability: resilience against accidental data loss or malicious modification of the data.
%	Finally, Blockchain systems can embed executable programs that can compute as appropriate and enjoy the benefits of Blockchain technology's other capabilities.
	
%	\item \textbf{Identifying groups of applications (i.e., use cases) that are most likely to benefit from Blockchain technology.}
%	Within the literature we analyzed there was a range of potential applications for Blockchain technology.
%	We group these applications into a set of use cases and then discuss the likelihood that individual applications within the use case would benefit from the use of Blockchain technology.
%	Example use cases include cryptocurrencies, asset management, and multi-organization data sharing.
%	
%	\item \textbf{Detailing challenges and limitations related to Blockchain technology.}
%	As part of our review of the literature, we identified several important challenges and limitations for Blockchain technology: scalability, smart contract correctness and dispute resolution, stapling of on-chain tokens to off-chain assets, key management, and regulation.
%	Many of these challenges represent important research questions with interesting potential for future research.
%	In this paper, we also survey the academic research that has already been conducted in regards to Blockchain technology.

%	\item \textbf{Describe how Blockchain technology fits in relation to other distributed databases.}
%	We provide a taxonomy and flow chart that describes how Blockchain technology relates to other distributed databases.
%	This taxonomy demonstrates that Blockchain technology is the most complex of the related distributed databases.
%	As such, we recommend that applications leverage Blockchain technology only if they need the shared governance, provenance, auditability and resilience provided by Blockchain technology.

%	\item \textbf{Leveraging grounded theory to analyze industrial literature while limiting research bias and separating hype from sound technical details.}
%	While there are significant benefits to analyzing industrial literature, there is also a significant amount of hype and imprecise language.
%	To address these limitations, we leveraged the grounded theory methodology~\cite{glaser1965constant,strauss1990basics,corbin1990grounded} (also known as the constant comparative method) to extract and separate valuable technical insights out from the hype and technical misunderstandings that permeate this body of work.
%	Based on this analysis of data from industry, interspersed with our own knowledge and a review of the academic literature, we can answer the three questions we identified while remaining grounded in the data we analyzed.
%	We were also able to shed light on industry's understandings and misunderstandings of Blockchain technology. 
%	
%	
%\end{enumerate}