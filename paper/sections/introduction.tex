% !TEX root = ../main.tex

\section{Introduction}
%\rnote{I think that we need to start with the notion of an authenticated data structure or hash chain used to establish the integrity of data, given the title of the paper.  What is the right reference?}
% Jeremy: The right reference is Haber and Stornetta's work on linked timestamping
% Jeremy: An alternative meta-citation would be Narayanan17 which traces the history
% Jeremy: I will eventually fix the first paragraph: Chaum's e-cash doesn't meet the standard definition provided, in particular it is a digital representation of cash issued by a central bank and is not independent of it any way... it just allows electronic payments

In 1982, David Chaum~\cite{Cha82} proposed using blind signatures to allow for untraceable electronic payments.
Later, Chaum, Fiat and Naor~\cite{chaum1988untraceable} expanded this idea into a fully fleshed out electronic cash system.
eCash was the first cryptocurrency---\ie ``a digital currency in which encryption techniques are used to regulate the generation of units of currency and verify the transfer of funds operating independently of a [nation-state-controlled] central bank.''\footnote{\url{https://en.oxforddictionaries.com/definition/cryptocurrency}}
However, early cryptocurrencies still required the presence of a central party to help manage the creation and transfer of funds.

Two decades after eCash was first proposed, an author using the pseudonym Satoshi Nakamoto wrote a white paper describing Bitcoin, a new decentralized cryptocurrency~\cite{Nak08}.
Unlike centralized cryptocurrencies which rely on a set of known entities to operate, Bitcoin uses a proof-of-work-based scheme~\cite{DN93,back1997partial} to allow the general public to maintain the system.
To incentivize public participation, Bitcoin pays participants (known as miners) for solving the proof-of-work puzzles.
While Bitcoin's building blocks were not novel, the composition of these properties into a single system was a meaningful contribution~\cite{Narayanan17} which led the cryptocurrency to become the first to achieve widespread popularity and usage.

At the peak of its popularity, Bitcoin's market capitalization reached \$835.69 billion (USD), though at the time of this writing it has seen a decrease in both its stature and market capitalization (\$67.21 billion).
Nevertheless, there remains significant interest in the question of whether the technology underlying Bitcoin---known as \emph{Blockchain technology} or {Blockchain} for short---could be used to build other interesting cryptocurrencies or distributed systems.
In particular, we have consistently heard researchers, business executives, and government leadership ask the following three questions related to Blockchain technology: (1) what exactly is Blockchain technology, (2) what capabilities does it provide, and (3) what are good applications for Blockchain technology.%, and (4) how does it relate to other approache distributed technologies (e.g., distributed databases).

% Jeremy: We invert the common approach: we don't study blockchain people looking at use-cases, we study the use case people looking at blockchain
To answer these questions, it is insufficient to consider academic sources alone.
While there have been several academic overviews written in this space, they have not addressed these general question regarding Blockchain technology, but rather surveyed particular systems or technical properties: Bitcoin~\cite{BMC+15,Narayanan17}, payment privacy~\cite{Conti17}, security and performance~\cite{Gervais16}, scalability~\cite{Croman16}, and consensus protocols~\cite{Bano17,garay2018consensus}.
In contrast, there has been significant work from non-academic sources (hereafter referred to as \emph{industry}) attempting to answer the questions as they search for applications and new product offerings that could be enabled by Blockchain technology.  Both freshly-formed and decades-old companies have developed answers to these questions.  Unfortunately, some of this work is fueled by hype and most industry documents are published without a peer-review process and associated rigor expected of academic literature.

To address these limitations, we analyzed literature produced by industry using grounded theory methodology~\cite{glaser1965constant,strauss1990basics,corbin1990grounded} (also known as the constant comparative method).
This allowed us to extract and separate valuable technical insights out from the hype and technical misunderstandings that permeate this body of work.
Based on this analysis of data from industry, interspersed with our own knowledge and a review of the academic literature, we are able to answer the three questions commonly asked in relation to Blockchain technology, as well as shed light on industry's understandings and misunderstandings of the technology. 

While many of the results in this work might not be surprising to some readers, our experience suggests that knowledge of Blockchain technology is neither universal nor complete within the security community.
Taken together, the results in this paper represent the most complete overview of Blockchain technology and its potential use cases available in a single work that we are aware of, and are intended to serve as an aid for researchers as they field questions related to Blockchain and as they explore whether Blockchain technology is relevant to their personal research areas. More concretely, our key contributions are:

\begin{enumerate}
	\item \textbf{Defining the key technical properties of Blockchain technology.}
	We identify three key properties for Blockchain technology: shared governance and operation, the use of an append-only ledger, and replication of data.
	Blockchain systems are governed and operated by a distributed and decentralized collective of entities that ensure that all operations are valid.
	These operations are recorded in an append-only ledger that provides full-system provenance.
	This ledger is also replicated by a set of governing parties, providing resilience against data loss and malicious modification of data.
	We also discuss associated technical properties that Blockchain makes use of.
	
	\item \textbf{Identifying the capabilities provided by Blockchain technology.}
 	Blockchain provides four important capabilities.
	Blockchain systems inherently track full-system provenance, and this allows the system to be audited by miners and potentially by other parties.
	Provenance is also useful in tracking assets monitored or managed by the Blockchain system (e.g., supply chain management, cryptocurrencies).
	Blockchain technology provides access control for the assets it monitors or manages and can allow operations on these assets to be pseudonymous.
	Provenance information and replication of that information among all miners provides a third important capability: resilience against accidental data loss or malicious modification of the data.
	Finally, Blockchain systems can embed executable programs that can compute as appropriate and enjoy the benefits of Blockchain technology's other capabilities.
	
	\item \textbf{Identifying groups of applications (i.e., use cases) that are most likely to benefit from Blockchain technology.}
	Within the literature we analyzed there was a range of potential applications for Blockchain technology.
	We group these applications into a set of use cases and then discuss the likelihood that individual applications within the use case would benefit from the use of Blockchain technology.
	Example use cases include cryptocurrencies, asset management, and multi-organization data sharing.
	
	%TODO: These need to be updated when I work through these sections and the discussion.
	\item \textbf{Detailing challenges and limitations related to Blockchain technology.}
	As part of our review of the literature, we identified several important challenges and limitations for Blockchain technology: scalability, smart contract correctness and dispute resolution, stapling of on-chain tokens to off-chain assets, key management, and regulation.
	Many of this challenges represent important research questions with interesting potential for future research.
%	In this paper, we also survey the academic research that has already been conducted in regards to Blockchain technology.

%	\item \textbf{Describe how Blockchain technology fits in relation to other distributed databases.}
%	We provide a taxonomy and flow chart that describes how Blockchain technology relates to other distributed databases.
%	This taxonomy demonstrates that Blockchain technology is the most complex of the related distributed databases.
%	As such, we recommend that applications leverage Blockchain technology only if they need the shared governance, provenance, auditability and resilience provided by Blockchain technology.

	
\end{enumerate}
