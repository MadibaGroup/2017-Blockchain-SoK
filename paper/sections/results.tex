% !TEX root = ../main.tex

\section{Technical Properties}
\label{sec:blockchain}

\begin{figure}
	\centering
	\includegraphics[page=2,width=\columnwidth]{figures/grounded-theory-main}
	\caption{Technical Properties for Blockchain Technology}
	\label{fig:technical-properties}
\end{figure}

The first broad category of blockchain concepts are technical properties, which we subdivide into three key groups: \textit{shared governance and operation}, \textit{verifiable state}, and 
\textit{resilience to data loss}. Figure~\ref{fig:technical-properties} shows the relationship between them.

\subsection{Shared Governance and Operation}
\label{sec:sharedgov}

%Before describing how shared governance and operation defines Blockchain technology, it is important to define both of these concepts.
%Governance is the process by which one or more entities (e.g., an individual, a company, a government agency) define how a system will operate---i.e., what operations are supported and how those operations function.
%Governance can be singular---a single entity (e.g., a developer, a company) defines the system---or it can be shared---multiple entities work together to define the system (e.g., IETF, RFC).
%Operation refers to the number of parties that actually operate a system.
%Operation can be singular---a single entity executes the system (e.g., a trusted third party)---or shared---multiple entities collectively operate the system (e.g., through the use of a consensus protocol).
%Note, distributed operation (i.e., operation on many machines) does not imply shared operation, as the owner/operator of all distributed nodes may still be a single entity.

Blockchain technology addresses the scenario in which a collection of entities (for example, individuals or companies) want to participate in a communal system but do not trust each other or any third party to operate the system single-handedly. By deciding on the system details (governance) and then deploying networked devices (referred to as \emph{miners}) to run the system, each entity can become assured of correct operation. If a small number of the miners become compromised (within bounds that are highly nuanced), the uncompromised miners can reject the malicious actions taken by the compromised miners and preserve the correct operation of the system. In this regard, blockchain technology provides \emph{diffused trust} where the collective of miners is trusted. This is often given the misnomer `trustlessness' --- trust still exists but has been diffused.

% SR: Transactions include both functions stored on the blockchain and those defined by the system.

Shared operation is enabled by the use of \emph{consensus protocols} which are used by the miners to agree upon which operations---known as \emph{transactions}---will be executed by the system. A transaction is sometimes what it sounds like, a financial transaction that moves a unit of value from one account to another, but more generally is a request that a certain function (which itself may be stored in the blockchain) be executed on a set of inputs given in the transaction. Shared governance exists over what valid transactions look like (\eg it is digitally signed by the sender) and how the system functions (\eg the size and number of operations in a transaction are less than a certain bound). Shared operation means every miner validates transactions, and consensus among miners is used to ensure that only correct outputs of valid transactions are written to the blockchain (invalid or incorrectly executed transactions can be proposed but will be rejected by the miners).

% JC: Omitting forks for now as this seems not that critical and require a lot of concepts to describe

%The consensus protocols allow miners to view transactions and validate that the 
%system is operating appropriately. This is an important consideration as it is not only valid transactions that will be submitted to the consensus protocol. Shared governance is possible because miners  only approve the transactions they believe are acceptable, according to hardcoded software rules, effectively allowing them to vote for how the system should function. If there is disagreement regarding system functionality, it is possible for a blockchain-based system to fork, resulting in the creation of competing systems that include only a subset of each other's transactions. Usually these forks are temporary, with miners eventually choosing to all adopt the same rules, but it is possible for a fork to result in the permanent creation of two non-interoperable blockchain systems (\eg Bitcoin and Bitcoin Cash).

%In Blockchain systems that use a majority-voting consensus mechanism for transaction validation, there are two types of forks that can occur.
%In a soft fork, transactions that validate with the modified rules will also validate with the original rules, but transactions that validate with the original rules might not validate with the modified rules.
%In a hard fork, transactions that validate with the modified rules will not necessarily validate with the original rules.
%The benefit of a soft fork is that both sets of miners can continue participating in the first consensus protocol, with transactions following the modified rules always being accepted and the transactions following the original rules only being accepted if there is a majority of miners who still use the original rules.
%While a soft fork is not a permanent situation, it can provide time for miners to slowly adopt the modified protocol while allowing both sets of miners to operate on the same data.

Blockchain systems can be categorized based on who is allowed to act as a miner:
%\footnote{In our coding, we also identified the concept of private governance, which eschews the notion of miners. This concept is discussed later in Section~\ref{sec:private-blockchain}.}

\begin{itemize}

	\item \emph{Open governance (i.e., permissionless blockchains).}
	Any party that is willing to participate in the consensus protocol is allowed to do so.
	Because identities systems generally require trusted third parties, open governance must address sybil attacks. This involves consensus protocols where miners prove ownership and/or expenditure of some costly, finite resource.
	Proof-of-work (demonstrating ownership of computing resources) and proof-of-stake (staking  digital assets owned on the blockchain system) are two common methods~\cite{Bano17,garay2018consensus}.
	
	\item \emph{Consortium governance (i.e., permissioned blockchains).}
	Participation in the consensus protocol is limited to miners approved on a whitelist, defined at system initialization.
	If this set never changes, it is known as a \emph{static consortium}.
	Alternatively, in an \emph{agile consortium} miners change over time, either based on the rules of the system (\eg random selection) or through consensus by the existing miners.
	Because each miner in a consortium is mapped to a known identity, a traditional byzantine fault tolerant protocol (from distributed systems) can be used. This sidesteps the wasteful resource expenditure of sybil-resistant protocols like proof of work~\cite{Bano17,garay2018consensus}.
	
\end{itemize}

For each type of governance, there is a need to incentivize correct participant behavior.
The first type of incentive is an \emph{intrinsic incentive}---i.e., miners maintain the system faithfully because they derive value from using it. Next, \emph{on-chain incentives} exist when the Blockchain system provides direct benefits to miners for faithful execution (e.g., minting currency and giving it to the miners). Finally, \emph{off-chain incentives} are any incentive that is not managed by the Blockchain system---for example, contractual obligations or individual reputation.
Importantly, off-chain incentives only apply to consortium governance as they inherently rely on knowing the identity of the miners.

%Note, the literature sometimes refers to open and consortium governance Blockchain systems as \emph{permissionless} and \emph{permissioned} Blockchains, respectively.
%We avoid this terminology as it can be confusing to readers.
%For example, an open governance Blockchain system can still implement permissions for how users can interact with assets tracked by the Blockchain; this is confusing if the system is also referred to as ``permissionless''.

\subsection{Verifiable State}

Entities adopt blockchain technology because they want their trust to be rooted in the system---i.e., that the current state of the system accurately reflects the transactions that the consensus protocol allowed to execute in the past.
To enable this trust, miners write all transactions to a cryptographically-verified append-only ledger~\cite{tamassia2003authenticated}, providing full system provenance and allowing miners (or outside parties) to audit the system's current state and past operations.
In many systems, including Bitcoin, this ledger is colloquially referred to as ``the blockchain'' (we will avoid this term as it unnecessarily confusing to discuss both holistic blockchain technology and the blockchain data structure). All transactions are strictly ordered and after consensus is reached (and as long as it is maintained), this ordering never changes and transactions are never removed. Thus All miners who begins at the first entry (called the \emph{genesis block}) will process all the transactions in the same order and reach the same current state for the entire system.

% JC: Below is quite Bitcoin-specific, Ethereum for example doesn't exactly work this way. It also seems to me to be unnecessary detail for the Queue version, so removing most of it for now.

%The first entry in the append-only ledger is known as the \emph{genesis block}.
%The genesis block is responsible for specifying the initial parameters for the system.
%New transactions are added to the ledger and cryptographically linked to one or more preceding transactions (or the genesis block for the first transaction)---for example, by signing the concatenation of the latest transaction and a hash of the transactions it is linked to.
%The resulting data structure can be either linear (e.g., Bitcoin's hash chain) or branching (e.g., a Merkle tree or directed acyclic graph).
%Regardless of the underlying structure, it is critical that all transactions are strictly ordered and that this ordering never changes after consensus is reached.

%Transactions stored in the append-only ledger can contain any data allowed by the consensus protocol, but in practice transactions are usually concerned with \emph{tokens}.
%Tokens represent a resource that is either on-chain (e.g., cryptocurrency, a document) or off-chain (e.g., a diamond, a file stored in the cloud) and are used to track that resource within the Blockchain system.
%For off-chain assets, there needs to be the ability to \emph{staple} the on-chain token to the off-chain assets.
%While there have been a variety of proposals for doing this (e.g., etching the token's identifier onto the physical items), effective stapling remains a challenge (see Section~\ref{sec:challenges}).

\subsection{Resilience to Data Loss}

If the ledger was only stored in a single location, deleting or modifying that data store could be detected by all parties, but there would be no guarantee that the data could be restored.
With blockchain technology, the contents of the ledger is replicated among all miners to address this single point of failure. When data does need to be restored---for example, if an individual miner's ledger is corrupted or a new miner joins---the replicated data can be verified to ensure that it correctly represents the system state.

%The cryptographically-authenticated append-only ledger is replicated among a set of miners and this replication is important for several reasons.
%First, during the consensus protocol it is necessary for miners to be aware of previous transactions that might invalidate the transaction being considered for approval.
%Second, it removes a single point of failure, preventing the loss of data at one site from impacting the overall system.
%Third, it protects against malicious attempts to modify the append-only ledger.
%Without replication it would still be possible to detect that data had been corrupted, but there is no guarantee that the original data could be restored.

%The use of a cryptographically authenticated ledger also has the benefit that it is possible to verify that the current state properly proceeds from the initial genesis block.
%While this is useful for runtime integrity checks, it is also useful when a miner needs to rebuild their state.
%In this case, they can replicate data from another miner and verify its authenticity without the need to trust the other miner as would be required in most distributed databases.

Some Blockchain systems try to limit the amount of data any given miner needs to replicate by segmenting the data and assigning miners to handle governance and operations for only a subset of the system.
This is known as \emph{sharding}, with individual segments of the data known as \emph{shards}.
Sharding can drastically reduce the amount of data that miners need to store while also increasing the performance of the consensus protocol, which often scale based on the number of miners.
Still, sharding adds complexity to auditing the system as a whole.
Additionally, by reducing the number of miners responsible for any given transaction, sharding also reduces the number of miners an adversary would need to deceive an end client about a transaction's existence.

\section{Capabilities}
\label{sec:capabilities}

\begin{figure}
	\centering
	\includegraphics[page=4,width=\columnwidth]{figures/grounded-theory-main}
	
	{\small Purple---capabilities, blue---technical properties, green---technical primitives. Arrows indicate that the destination depends on the source.}
	\caption{Capabilities for Blockchain Technology}
	\label{fig:Capabilities}
\end{figure}

Capabilities define the high-level functionality that can be achieved by using Blockchain technology in a system's design.
Blockchain's three core capabilities were described in the preceding section: (1) shared governance and operation, (2) verifiable state, and (3) resilience to data loss.
In our coding, we identified 11 additional capabilities (see Figure~\ref{fig:Capabilities}).

\paragraph{Provenance and Auditability}
Blockchain systems provide a complete history of all transactions that were approved by the consensus process (i.e., full-system provenance).
%While failed transactions are not normally written to the ledger, they could be if needed by the system for auditability.
This information can be used by the miners to audit the system and ensure that it has always followed the appropriate rules.
Additionally, this information can be used by non-miners to verify that the system is being governed and operated correctly.

If transactions are used to store information regarding digital or real-world resources, the resources must be \emph{stapled} to on-chain identifiers. Then the provenance information for the Blockchain system can also be used to provide audit information for those resources.
This can be used to track physical, off-chain assets (e.g., for supply chain management), digital, off-chain assets (e.g., copyrighted digital media), or digital, on-chain assets (e.g., cryptocurrencies or data files).
 
\paragraph{Access Control and Pseudonymity}
% JC: I find this section confusing still... :(
% SR: This is over simplified somewhat and coule be improved for the full version, but is inline with other simplifications in the Queue version.
Data stored in a blockchain may have limitations regarding which users can use it as an input to a transaction or modify it as part of the operation of the transaction.
For example, a financial asset should only be a valid input to a transaction if the owner of that asset approves its use.
One approach to providing this functionality is storing access control lists (ACLs) in the blockchain and having the appropriate users prove their identity to the miners (e.g., using Kerberos or OAuth 2.0) as part of the transaction validation process.
More commonly, access control in a blockchain system is implemented cryptographically: data is associated with a public key when it is created and the ability to use or modify this data as part of transaction is granted only to users that can prove knowledge of the corresponding private key (e.g., by generating a signature over the transaction that validates with the public key attached to the data).
Ownership of the data can be expanded or transferred by associating it with a new public key.

%On key benefit of access control using Blockchain is that the provenance of access control is automatically recorded.
%This means that a full record of not only a user's permissions, but how they received those permissions, is stored.
%This information can be used to automatically revoke permissions if it is discovered that a user was granted these permissions by a compromised account---for example, when a malicious insider grants inappropriate permissions to other insiders.

Key-based (as opposed to ACL-based) ownership of data has another advantage: it allows for pseudonymous ownership and use of data.
Still, this requires careful attention in the system design to use appropriate cryptographic techniques (e.g., zero-knowledge proofs, mix networks, or secure multi-party computation) to avoid linking real-world individuals to their keys and actions. This remains an open problem.

\paragraph{Automatic Execution (Smart Contracts)}
In a general purpose blockchain, smart contracts or decentralized applications (DApps) are deployed with a transaction that stores the code for a set of functions and the initial state of the contract. These functions can then be called in subsequent transactions. The functions themselves are executed by the miners, and outputs are verified through the consensus protocol. Any entity can execute any function but the function itself might be programmed to fail if the conditions under which it is called are not what the designer intends. The computational power of the scripting language that can be used to specify a function varies from blockchain to blockchain, and there are  many nuances to ensure functions can be executed by each miner deterministically in a timely fashion. Bitcoin is known for a highly limited scripting language that enables little beyond financial transactions, while Ethereum strives for highly verbose code capable of general computation.

%Smart contracts benefit from Blockchain technology's other capabilities (e.g., shared operation, auditability, and resilience).
%For example, multiple miners execute and verify the output of a smart contract to help ensure that an adversary is unable to tamper with the result of a function.
%Similarly, the ability to audit inputs and outputs can be used to attribute incorrect usage of a smart contract.
%Still, smart contracts suffer from problems common to all programs (e.g., bugs, security flaws, complexity, or non-termination) and a failure to recognize this reality can lead to disastrous consequences.\footnote{This is best exemplified by the debate over ``code is law'' and the DAO attack: \url{https://www.coindesk.com/understanding-dao-hack-journalists/}.}

\paragraph{Data Discoverability}
If users are allowed to read any record stored in a blockchain, then it is possible to search for records of interest.
This capability is nothing more than what is provided by having a read-only data lake, but still it was frequently discussed in the literature we reviewed.