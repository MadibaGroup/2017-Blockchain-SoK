% !TEX root = ../main.tex

\section{What is Blockchain Technology?}
\label{sec:blockchain}
Our description of Blockchain technology is based on the technical properties identified during our analysis (see Figure~\ref{fig:technical-properties}).
This analysis revealed three key groups of properties: shared governance and operation, a cryptographically-authenticated append-only ledger, and resilience.
By themselves, these components are nothing new, but used together they form what is known as \emph{Blockchain technology} or \emph{Blockchain} for short.

\begin{figure}
	\centering
	\includegraphics[width=\columnwidth]{figures/technical-properties}	
	\caption{Technical Properties for Blockchain Technology.}
	\label{fig:technical-properties}
\end{figure}

\begin{figure}
	\centering
	\includegraphics[width=\columnwidth]{figures/capabilities}	
	\caption{Capabilities for Blockchain Technology.}
	\label{fig:capabilities}
\end{figure}

\begin{figure}
	\centering
	\includegraphics[width=\columnwidth]{figures/normative-properties}	
	\caption{Normative Properties for Blockchain Technology.}
	\label{fig:normative-properties}
\end{figure}

\subsection{Shared Governance and Operation}
At the heart of Blockchain technology is the principle of shared governance and operation.
Either of these properties is common by itself; for example, having multiple parties govern how a system should function, but then relying on trusted third-parties to operate and maintain the system.
Alternatively, the literature is rife with well-defined systems that don't need ongoing governance, but require multiple parties to actually operate and maintain the system.

Blockchain is distinct, though not necessarily unique, in that it requires a core set of participants---referred to hereafter as \emph{miners}---who are responsible for both deciding how the system should function (i.e., shared governance) and then for operating that system.
This type of shared governance is appropriate when miners are not able to sufficiently trust each other or a third-party to faithfully govern and operate a system.
By participating in all aspects of governance and operation, each miner can be assured that the system is operating as intended.
Even if some of the miners are compromised, the other miners retain the ability to detect malicious actions by the compromised miner and to prevent it from affecting the system.
In this regard, Blockchain technology provides \emph{diffused trust} wherein it is not individual miners but rather the collective of all miners that is trusted.\footnote{This property has incorrectly been called ``trustlessness''. This is incorrect as trust still exists, it has just been diffused amongst multiple parties.} 

This shared governance and operation is executed using one or more \emph{consensus protocols} (e.g., proof-of-work~\cite{DN93,back1997partial,NakamotoS8}, byzantine fault tolerance~\cite{castro1999practical}).
The first consensus protocol is used by miners to determine what operations---known as \emph{transactions}---will be allowed to alter the state of the Blockchain system.
The second consensus protocol is used by miners to determine the rules that will be used to validate transactions in the first consensus protocol.
While governance could be conducted and recorded in transactions, removing the need for the second consensus protocol, we are not aware of any Blockchain systems that do so.

In practice, this second consensus protocol is usually an informal process in which changes to the first consensus protocol are discussed in a secondary channel (e.g., on an Internet discussion board), and consensus is established based on the number of miners that adopt the modified rules.
This ad-hoc consensus mechanisms means that it is possible for a Blockchain system to split, with one system being run by the set of miners continuing to operate using the original rules, and the other system being run by set of miners using the new rules.
Such a split is known as a \emph{fork}.
Often these forks are temporary, with miners either choosing to all adopt the new rules or to return to the original rules, but it is possible for a fork to result in the permanent creation of two non-interoperable Blockchain systems (e.g., Bitcoin Classic and Bitcoin Cash).

In Blockchain systems that use a majority-voting consensus mechanism for transaction validation, there are two types of forks that can occur.
In a soft fork, transactions that validate with the modified rules will also validate with the original rules, but transactions that validate with the original rules might not validate with the modified rules.
In a hard fork, transactions that validate with the modified rules will not necessarily validate with the original rules.
The benefit of a soft fork is that both sets of miners can continue participating in the first consensus protocol, with transactions following the modified rules always being accepted and the transactions following the original rules only being accepted if there is a majority of miners who still use the original rules.
While a soft-fork is not a permanent situation, it can provide time for miners to slowly adopt the modified protocol while allowing both sets of miners to operate on the same data.

Blockchain systems can be separated based on how they select who can act as miner:

\begin{itemize}
	\item \emph{Open governance.}
	Any party that is willing to participate in the consensus protocol is allowed to do so.
	As such, these systems are susceptible to Sybil attacks and it is necessary for them to use consensus protocols that rely on miners  proving ownership of some resource rather than relying on the miner's identity.
	Proof-of-work (demonstrating ownership of computing resources) and proof-of-stake (demonstrating ownership of digital assets stored by the Blockchain system) are the most common methods~\cite{Bano17,garay2018consensus}.
	
	\item \emph{Consortium governance.}
	Only approved miners that can attest to their identity are allowed to participate in the consensus protocol.
	The initial set of approved miners is defined at system initialization.
	If membership in the group of miners remains static over the lifetime of the system it is known as a \emph{static consortium}.
	Alternatively, in an \emph{agile consortium} miners change over time, either based on the rules of the system (e.g., random selection) or through consensus by the existing miners.
	Because miners in a consortium have a known identity they can use Byzantine Fault Tolerant consensus protocols, which do not require the resource expenditure of the Sybil-resistant protocols used in open governance-based systems~\cite{Bano17,garay2018consensus}.		
\end{itemize}

For each type of governance, there is a need to incentivize correct participant behavior.
The first type of incentive is an \emph{intrinsic incentive}---i.e., miners maintain the system faithfully because they derive value from using it.
Next, \emph{on-chain incentives} are when the Blockchain system provides direct benefits to miners for faithfully executing the system (e.g., minting currency and giving it to the miners).
Finally, \emph{off-chain incentives} are any incentive that is not managed by the Blockchain system---for example, contractual obligations or reputation.
Importantly, off-chain incentives only apply to consortium governance as they inherently rely on knowing the identity of the miners.

\subsection{Cryptographic Append-Only Ledger}
While a Blockchain system might store its current state for convenience and performance, this is not actually a requirement in Blockchain technology.
Instead, the key data structure in Blockchain technology is a cryptographically authenticated data structure~\cite{tamassia2003authenticated} that stores a history of all the transactions that have been approved by the miners.
This \emph{ledger} provides full system provenance and allows for miners or other outside parties to audit the system.
In Bitcoin, this ledger is colloquially referred to as the ``blockchain'', but we avoid that term as it unnecessarily confusing to try and discuss both Blockchain (big-B) technology and the blockchain (little-b) data structure.

The first item in the append-only ledger is known as the \emph{genesis block}.
The genesis block is responsible for identifying the initial parameters for the system.
Whenever a new transaction is approved by the miners, it will be added to the ledger and cryptographically linked to one or more preceding transactions (or the genesis block for the first transaction)~\cite{bayer1993improving,haber1990time,haber1997secure}---for example, by signing a combination of the latest transaction and a hash of the transactions it is linked to.
The resulting data structure can be either linear (e.g., Bitcoin's hash chain) or branching (e.g., Merkle tree, directed acyclic graph).
Regardless of the underlying structure it is critical that all transactions are strictly ordered and that this ordering never changes.

Transactions stored in the append-only ledger can contain any data allowed by the consensus protocol, but in practice transactions are usually concerned with \emph{tokens}.
Tokens represent a resource that is either on-chain (e.g., cryptocurrency, a document) or off-chain (e.g., a diamond, a file stored in the cloud) and are used to track that resource within the Blockchain system.
For off-chain assets, there needs to be the ability to \emph{staple} the on-chain token to the off-chain assets.
While there have been a variety of proposals for doing this (e.g., etching the token's identifier onto the physical items), effective stapling remains an open research challenge.

\subsection{Resilience}
The cryptographically-authenticated append-only ledger is replicated amongst all miners and this repplication several reasons.
First, during the consensus protocol it is necessary for miners to be aware of previous transactions that might invalidate the transaction being considered for approval.
Second, it removes a single point of failure preventing the loss of data at one site from impacting the system.
Third, it protects against malicious attempts to modify the append-only ledger. Without replication it would still be possible to detect that data had been corrupted, but without the replication there is no guarantee that it could be restored.

Some Blockchain systems try to limit the amount of data any given miners need to replicate by segmenting the data and assigning miners to handle governance and operations for only a subset of the system.
This is known as \emph{sharding}, with individual segments of the data known as \emph{shards}.
Sharding can drastically reduce the amount of data that miners need to store while also increasing the performance of the consensus protocols which often scale based on the number of miners.
Still, sharding comes with the drawback that miners are no longer able to audit the system as a whole.
Additionally, by reducing the number of miners responsible for any given transaction, it also reduces the number of miners an adversary would need to compromise to attack a given shard.

The use of a consensus protocol and data protection provides significant resilience to Blockchain systems.
Most importantly, the compromise of a single miner, or even a small number of miners, will not impact the functionality of the system.
If the compromised miner attempts to include fraudulent transactions, these transactions will be detected and rejected by the other consensus partners.
These partners can even alert the compromised party that they have been acting oddly and might be compromised.

If a miner's data is lost or corrupted---either accidentally or maliciously---that miner will be able to restore their copy of data by replicating it from other miners.
In such cases, the miner recovering from data loss is responsible for verifying that the genesis block is correct and that each successive transaction follows the established rules, including his participation as a miner.
In this way, rebuilding a data store represents diffused trust in the system and not individual trust in any given miner.


\section{Blockchain Technology's Capabilities}
\label{sec:capabilities}

Blockchain technology assembles its various technical primitive and properties to provide several key capabilities.

\subsection{Provenance and Auditability}
Blockchain systems provide a complete history of all transactions that were approved by the consensus process (i.e., full-system provenance).
While failed transactions are not normally written to the ledger, they could be if needed by the system for auditability.
This information can be used by the miners to audit the system and ensure that it has always followed the appropriate rules.
Additionally, this information can be used by non-miners to verify that the system is being governed and operated correctly.

If transactions are used to store information regarding digital or real-world resources (using tokens), then the provenance information for the Blockchain system can also be used to provide audit information for those resources.
This can be used to track physical, off-chain assets (e.g., supply chain management, tracking diamonds), digital, off-chain assets (e.g., copyrighted digital media), or digital, on-chain assets (e.g., cryptocurrencies, files).
 
\subsection{Access Control and Anonymity}
Whether a user of a Blockchain system is able to create, update, or delete a token is based on permissions defined by the miners.
While these permissions could be tracked using traditional access control paradigms, most often they are regulated cryptographically.
In this paradigm, when a token is created it is also associated with a public key.
The ability to update or delete this token is then granted to any users that can prove knowledge of the corresponding private key (e.g., by generating a signature that validates with the public key attached to the token).
Ownership of the token can be transferred or shared by associating it with a new public key.

The biggest challenge towards key-based ownership of tokens is the need to manage a public key infrastructure (PKI).
This is both a hassle technically~\cite{CT} as well as for users~\cite{ruoti2015johnny,barber2012bitter}.
One advantage of having key-based, not user-based ownership of tokens is that it allows for anonymity in the ownership and use of tokens.
Still, this requires careful attention in the system design to use appropriate cryptographic techniques (e.g., zero-knowledge proofs, mix networks, secure multi-party computation) to avoid linking real-world individuals to their keys and actions.

\subsection{Smart Contracts}
Blockchain tokens can also represent and store executable functions known as \emph{smart contracts}.
These smart contracts can be executed automatically in response to a function call in later transactions, with both the inputs and outputs of the function recorded within the calling transaction.
The smart contracts themselves are executed by the miners with outputs being verified through the consensus protocol.
The computational power of these scripts is determined is the system's rules, ranging from supporting only basic functionality (e.g., verifying a signature in Bitcoin) to providing Turing-complete functionality (e.g., Etherium).

Smart contracts benefit from Blockchain technology's other capabilities (e.g., shared operation, auditability, and resilience).
For example, multiple miners execute and verify the output of a smart contract to help ensure that an adversary is unable to tamper with the result of a function.
Similarly, the ability to audit inputs and outputs can be used to attribute incorrect usage of a smart contract.
Still, smart contracts suffer from problems common to all programs (e.g., bugs, security flaws, complexity, non-termination) and a failure to recognize this reality can lead to disastrous consequences.\footnote{This is best exemplified by the debate over ``code is law'' and the DAO attack: \url{https://www.coindesk.com/understanding-dao-hack-journalists/}.}