% !TEX root = ../main.tex

% Jeremy: I reclustered this a bit. I think we discussed this during the call. Feel free to debate. 
% If you want to play around with the clusters: https://drive.google.com/file/d/1gDVd4x-SN6-QHZe52toEzMWXnLfiNtqW/view?usp=sharing
% I feel like separating challenges and limitations is a bit of a judgement call. 

%Our concept map shows interconnections between features and use cases. We also coded challenges -- both problems hindering the use of Blockchain that do not currently have satisfactory solutions, and inherent deficits of the technology. In this section, we will list the concept groupings we created and we discuss the academic response in the next section.
%\anote{I really like this list.  But, the challenges in Section~\ref{sec:challenges} only cover a subset of the list (and its not divided in the same way).  Should we aim to cover all the challenges listed here (for many of them, I dont really know what to say)?  We should definitely get the two sections to match up better.}
%\subsubsection{Technical Challenges}
%\begin{itemize}
%	\item{Blockchain can handle finite, countable, and unique resources only}
%	\item{Blockchain is not a high performance system}
%	\item{Lack of API access means auditors must run full nodes}
%	\item{Interoperability: cryptocurrency fragmentation; existence of too many implementations; siloed solutions; standardization; risks to overlay assets if underlying assets are mishandled; interfaces with existing institutions and systems; user identification across systems; risks posed by sharing blockchain security through merged mining or anchoring}	
%	\item{Susceptibility to coordinated attacks by large parties}
%	\item{Off-chain functionality: off-chain program execution; interoperability with off-chain systems}
%	\item{Privacy: anonymity; confidentiality}
%	\item{Resilience: distributed denial-of-service (DDOS) attacks; dishonest majority attacks; security of infrastructure; subversion of software security measures}		
%	\item{Scalability: block generation frequency; block size limits; computational cost of public blockchains; scalable and secure end-user software}
%	\item{Security}
%	\item{Smart contract correctness: inherent incompleteness of contracts; ensuring completeness; lack of tools for verification}
%\end{itemize}
%
%\subsubsection{Socio-Technical Challenges}
%\begin{itemize}
%	\item{Blockchain is an inefficient use of computing resources}
%	\item{Cryptocurrency economics: illiquidity, price volatility, high initial adoption costs, currency conversion costs, and decreasing marginal returns for miners}
%	\item{Incentives: correctly configuring game-theoretic incentive structures}
%	\item{Key management: difficulty of manual key management; possible unrecoverable loss of private keys; attacks against wallets; bugs and glitches in wallets}
% Text from a different section: The biggest challenge towards key-based ownership of tokens is the need to manage a public key infrastructure (PKI).
%This is both a hassle technically~\cite{CT} as well as for users~\cite{ruoti2015johnny,barber2012bitter}.
%	\item{Lack of protection against mistakes: transactions cannot be reversed; administrators cannot restore access if users are locked out}
%	\item{Miner centralization}
%	\item{Usability: difficulty of developing distributed apps; inherent complexity of technology; difficulty of access and use by consumers; education; onboarding users; difficulty of search; poor UX; lack of mobile and web clients}
%\end{itemize}
%
%\subsubsection{Challenges to Market Viability}
%\begin{itemize}
%	\item{Efficiency and cost: low or bottlenecked throughput; wasteful energy consumption; high transaction fees; latency induced by synchronous communication required by certain consensus protocols}
%	\item{Expense of developing end-user applications for individual blockchains}
%	\item{Usefulness: few demonstrable use cases; does not improve upon existing solutions in many domains being pursued}
%	\item{Unnecessary: if a central party is required; if a trusted intermediary exists; if a small number of parties are involved in the system}
%\end{itemize}
%
%\subsubsection{Use Case Challenges}
%\begin{itemize}
%	\item{Binding digital entities to real-world entities: stapling tokens to assets; interoperating with existing systems (e.g., compatibility between cryptocurrencies and cash)}
%	\item{Dispute resolution: difficulty recovering from errors or bugs; difficulty reversing fraudulent transactions; non-applicability to scenarios where a central broker is needed}
%\end{itemize}
%
%\subsubsection{Regulatory Challenges}
%\begin{itemize}
%	\item{Governance: resolution of conflicts requiring external intervention; incident response; rule updates require forks; transparency of software development}
%	\item{No distinct legal framework}
%	\item{Standardization}
%	\item{Regulatory: anti-money laundering, know-your-customer; difficulty of monitoring; lack of regulation; legal considerations; taxation; exchange control and flow management; consumer protection}	
%	\item{Cryptocurrencies are inherently Ponzi schemes}
%	\item{Reputation: use for crime; associations with black markets; nebulous or illicit uses; terrorist financing}
%\end{itemize}

% Jeremy: Arkady -- please see reclustering of previous section and use it to motivate why these are the right subset to study.
% Is this subset of challenges normative or positive?
% Normative -- the set of challenges that ought to be studied
% Positive -- the set of challenges that happened to be studied by others
\section{Key Challenges for Blockchain Technology}
\label{sec:challenges}

Our analysis of the literature revealed several challenges that need to be considered when developing systems using Blockchain technology.
%In this section, we describe these challenges and limitations.
%In Appendix~\ref{sec:academic}, we survey academic research that is attempting to address some of the challenges.
%Note that this section is focused on challenges facing Blockchain systems generally, not challenges facing specific applications such as Bitcoin or Ethereum).
\snote{Add reference to technical report for academic survey?}

\subsection{Scalability and Performance}
Decentralized governance and operation incur three forms of overhead: (1) the 
need to run a consensus protocol before state can be updated, (2) the need to 
store the full system provenance, and (3) the need for each miner to store the 
ledger in its entirety. %Any attempt to avoid this replication comes at the 
%cost of the resilience that Blockchain-based systems often rely upon.
Additionally, most of today's open governance Blockchain systems are based on 
proof-of-work, which brings additional challenges.
Users must acquire hardware and expend electricity to participate in consensus, 
the real-world cost of which can be tremendous---for example, it was estimated 
that as of April 2018 the energy consumed by Bitcoin miners alone was 
equivalent to the power usage of almost 5.5 million US 
households~\cite{Digiconomist}.

%Another unintended consequence of proof-of-work is the centralization of mining power.  In order to reduce variance in their earnings, miners are incentivized to work together in large mining pools, pooling their computing power and sharing the profits among pool members. This phenomenon has emerged in all large-scale proof-of-work Blockchain systems\footnote{Currently almost 70\% of Bitcoin blocks are mined by the five largest mining pools~\cite{BlockchainInfoPools}}, and it is a problem because it inhibits decentralization~\cite{arxiv:GBERS18}.
%Of course, these pools are disincentivized to damage trust in the system (and thus reduce its value and their profits) by abusing their power to censor transactions or violate rules in other ways. But this centralization undoubtedly runs counter to the normative property that Blockchain is decentralized and it may violate security notions that depend on decentralization.  

%Additionally, many design choices limit the performance of Blockchain systems.  Small block sizes can lead to extremely high transaction fees when miners are unwilling to add transactions that result in having to create more blocks.  On the other hand, larger block sizes can impede  throughput due to the increased latency of finalizing a transaction. %In Bitcoin, one must wait for six additional blocks to be confirmed before a transaction is considered final, which may be prohibitively long for applications where transactions need to clear immediately.
%The challenge of handling a huge number of transactions while achieving acceptable levels of throughput and latency remains a critical hurdle to the use of Blockchain solutions in performance-critical systems.

\subsection{On-chain Correctness}
All executable code is subject to bugs, and smart contracts are no exception.
The immutability of a Blockchain's ledger exacerbates this challenge by impeding rollback of state changes, even those that are clearly malicious.
Failure to act can be costly (e.g., the DAO attack~\cite{dao}), but so too can reversing transactions.
%This is because, by definition, any transactions on a blockchain upon which consensus is reached are considered legal---including ones due to buggy code and exploitations of such.
If miners decide to roll back the ledger to erase a mistaken transaction,  
confidence in the Blockchain system may be lost. The rollback system itself 
must be designed carefully, or there is risk of further 
exploitation~\cite{arxiv:AviSafSha18}.
Alternatively if miners can't agree what to do about errant transactions, it 
could lead to a fork in the Blockchain system.

%Despite best efforts to eliminate mistakes in smart contract and transactions, a payment or asset transfer system must be able to reverse fraudulent or errant transactions. For example, if a user pays for a physical real-world good using a cryptocurrency but is then never given the purchased item. A new transaction reversing the effects of the disputed transaction could be added to the ledger, but decentralized governance makes arbitrating such a dispute difficult as there is no individual arbiter with the authority to determine which party is in the right when a dispute occurs.  Additionally, dispute resolution must be handled carefully to avoid introducing new vulnerabilities.  For example, several attacks were demonstrated against the Bitcoin refund mechanism~\cite{FC:MccShaHao16} necessitating further research to design secure refunds in Bitcoin~\cite{arxiv:AviSafSha18}.

\subsection{Off-chain stapling}
Many Blockchain applications manage off-chain assets by representing them 
on-chain using digital identifiers, or \textit{tokens}. A major challenge for 
these application is ensuring consistency between on-chain state and the 
off-chain reality it represents.
When dealing with digital assets, consistency can be maintained by code; for 
example, a smart contract can track transference of ownership for a digital 
media license.
For physical assets, real-world processes must be employed to ensure 
consistency. These processes are an obvious point of failure as they rely on 
correct execution by trusted parties (something that Blockchain systems are 
often deployed to remove).
%For example, a malicious entity could attach two tokens to one asset, two assets to one token, or issue tokens that have no backing asset (e.g. stocks in a naked short selling scenario).
The end user must also be trusted, as they may be able to separate the token and sell it while keeping the asset, causing the token to be attached to an invalid asset (e.g., fake goods in luxury markets).

Similar challenges arise when Blockchain systems must track real-world events and information (e.g., sports scores, web requests).
While such information can be provided by \emph{off-chain oracles}, these are trusted entities that are difficult to audit.

%A related issue arises when Blockchain processes and smart contracts need to use off-chain inputs. 
%For example, gambling contracts may determine which address to pay winnings to based on the result of a request to an off-chain oracle (e.g., sports scores, stock prices, weather forecasts, or other global events).
%If contract logic branches based on that response, the contract is no longer verifiable because auditors cannot confirm that the response received from the oracle at audit time is the same response received when the contract was executed. There are legitimate reasons why an oracle response might change with time, so this is really an inherent limitation of Blockchain technology: smart contracts cannot ``see'' external events.  Thus, additional mechanisms are needed to verify that these inputs cannot be forged.

\subsection{Security}
Due to its decentralized nature, Blockchain is potentially vulnerable to a number of security threats.
Coordinated attacks by a majority (or often even a large minority) of the miners can reorder, remove, and change transactions from the ledger. 
%Thus, it is critical that Blockchain applications provide the proper incentives to keep miners honest.
% In particular, it is critical to design incentives such that the game theoretic behavior for selfish miners is to honestly maintain the state of the Blockchain system rather than to destroy it through forming such coalitions.
Additionally, Blockchain technology is vulnerable to traditional network attacks such as denial of service or partitioning.
Such attacks aim to lower the number of participating miners or to fracture the network of miners to prevent consensus, lower the bar for attacks, or create an inconsistent state.

\subsection{Privacy and anonymity}
%Another major challenge is how to protect the privacy of the users and data stored on a Blockchain ledger.
Data in a Blockchain ledger is public (at least to all miners) in order to enable verification, meaning that sensitive data is inherently non-private.
Confidentiality can be provided by hosting a Blockchain system where only trusted entities can access it or by applying advanced cryptographic techniques that allow miners to verify the correctness of encrypted transactions (e.g., zero-knowledge proofs, secure multi-party computation, and functional encryption)~\cite{SP:KMSWP16}.
Still, the use of cryptography limits auditability and the ability to have 
meaningful shared governance.

Extreme care must be taken when trying to build an anonymous Blockchain system.
While many existing Blockchain systems provide a notion of ``pseudonymity'' in which users are identified by their cryptographic keys instead of by their real-world names, this does not provide true anonymity as attacks that correlate transactions by the same pseudonyms together with other data external to the Blockchain system can effectively deanonymize users~\cite{DBLP:journals/corr/abs-1708-04748}.

\subsection{Usability} %User Experience}
% Jeremy: I toned this waaaaaay down. Ethereum has excellent tools now: linters, debuggers, static analysis tools. Even the etheruem wallet can compile and do basic debugging of Solidity code. However, beyond Ethereum things get messy and so that is the new point. 
%For developers, development and analysis tools are critical to building secure applications in any domain.
The availability of user-friendly developer tools varies significantly 
depending on the maturity of the Blockchain platform. Some projects like 
Ethereum have mature tools, while others have very little support. Many 
Blockchain platforms are currently geared towards expert users and lack the 
user experience-focused tools needed to allow for easier use by non-experts.
A related challenge is that some Blockchain systems require users to store, 
manage, and secure cryptographic keys, which is known to be a significant 
impediment for most users~\cite{ruoti2019usability}.
% ~\cite{uss:WhiTyg99}.  A survey by Eskandari et al.~\cite{EBSC15} outlines these challenges as well as potential solutions for managing keys for Bitcoin.  They discuss solutions such as password-protected and password-derived keys as well as offline and air-gapped storage of the keys.  But, as the authors state all of these solutions have their drawbacks.

%With a key management solution in place, a further challenge is obtaining transactions for auditing purposes---auditing is a much touted capability of blockchain across our dataset. In a decentralized environment, running a full node on the network, or relying on a entity that does (such as a blockchain explorer website or a lightweight user-client) is necessary. 

\subsection{Legality and Regulation}
Some of the benefits claimed by Blockchain systems are not derived from the 
underlying technology, but rather from a lack of regulation compared to 
competing services. 
For example, Blockchain-based money transfer services are generally orders of 
magnitude faster than similar bank-provided services, but this speed increase 
does not arise from the use of Blockchain (which generally has higher technical 
overhead) but because the Blockchain-based services are not complying with the 
time-consuming regulatory processes that are required of the bank-provided services (e.g., 
mandatory holding periods).
Our analysis revealed widespread concern with regulatory issues surrounding 
cryptocurrencies, Blockchain-based assets, and other Blockchain applications. 
This is unstable from a risk perspective and will be untenable in the long run 
as regulators catch up to these schemes.

%It is important to note that regulation applies indirectly to technology, based on how the technology is used within a firm's operations.
%Therefore, there will be no direct regulation of Bitcoin, for example, but rather regulation of firms that use Bitcoin according to how they are using it. 
%Consider the example of a Bitcoin exchange service that derives income from fees paid in Bitcoin: it will have to consider tax declarations as a business (\eg service taxes and capital gains), financial reporting as a money service business (\eg know your customer, anti-money laundering and anti-terrorist financing), generally acceptable accounting standards for audited financial reports (\eg reporting Bitcoin as an intangible asset on a balance sheet), and potentially additional regulation that applies to financial exchanges, banks, and/or custodians.
%In most countries, each of these already broad categories are administered by different government branches.
%Countries like the United States and Canada may require licensing or registration and have taken enforcement action against non-compliant firms.
%An extreme case of regulation is prohibition of cryptocurrencies or Blockchain-assets. At the time of writing, the largest country to ban Bitcoin is Pakistan and the largest country to prohibit wide categories of cryptocurrency use is China.
 
 
%As applications of blockchain technology proliferate, they have drawn significant attention from the regulatory bodies around the world.  In the settings of cryptocurrencies, a number of concerns such as the prevalence of black-market transactions, tax evasion, money laundering, and terrorist financing have drawn calls to regulate how such cryptocurrencies can be used.  An excellent review article by Kiviat~\cite{Kiviat15} outlines some of the issues that arise in regulating blockchain transactions.  

% Jeremy: I like this paper but it is a bit esoteric to real world concerns:
%Additionally, as blockchain applications move to greater support of smart contracts, researchers have shown that criminal smart contracts are easily implementable in today's smart contract platforms~\cite{CCS:JueKosShi16} requiring regulation to avoid such criminal uses of blockchain.  

% Jeremy: This is too speculative in my opinion and there is enough to talk about otherwise:
%Another regulatory issue that has recently come to light is whether blockchain technology is compatible with European Union's General Data Protection Regulation (GDPR) that has recently been put into law in the EU.  GDPR requires that individuals have the ``right to be forgotten'', i.e., that they should be able to expunge outdated or invalid information about themselves from available records.  However, since any data stored on a blockchain is immutable, this seems to pose a challenge for GDPR compliance.  Much research within the technical and regulatory communities is needed to ensure that blockchain can comply with existing regulation and to prevent the many nefarious uses of this technology.

%\subsection{Necessity of blockchain}
%Finally, before a company chooses to deploy Blockchain it should carefully consider whether it is the right tool for the job.  Due to the hype around this technology many solutions have turned to Blockchain when other, more established technologies such as traditional databases would do the job. For example, in situations when a centralized repository is easily available, it is likely to be preferable to a blockchain.  In fact there are relatively few use cases where blockchain is demonstratively the ``right'' solution.  While we believe that more such use cases will be identified, when evaluating a new use case it is important to make sure that Blockchain is really needed and that a lower-overhead solution will not suffice.