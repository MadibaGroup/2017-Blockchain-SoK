% !TEX root = ../main.tex

\section{Research Challenges}
\label{sec:challenges}

% Jeremy: I reclustered this a bit. I think we discussed this during the call. Feel free to debate. 
% If you want to play around with the clusters: https://drive.google.com/file/d/1gDVd4x-SN6-QHZe52toEzMWXnLfiNtqW/view?usp=sharing
% I feel like separating challenges and limitations is a bit of a judgement call. 

%Our concept map shows interconnections between features and use cases. We also coded challenges -- both problems hindering the use of Blockchain that do not currently have satisfactory solutions, and inherent deficits of the technology. In this section, we will list the concept groupings we created and we discuss the academic response in the next section.
%\anote{I really like this list.  But, the challenges in Section~\ref{sec:challenges} only cover a subset of the list (and its not divided in the same way).  Should we aim to cover all the challenges listed here (for many of them, I dont really know what to say)?  We should definitely get the two sections to match up better.}
%\subsubsection{Technical Challenges}
%\begin{itemize}
%	\item{Blockchain can handle finite, countable, and unique resources only}
%	\item{Blockchain is not a high performance system}
%	\item{Lack of API access means auditors must run full nodes}
%	\item{Interoperability: cryptocurrency fragmentation; existence of too many implementations; siloed solutions; standardization; risks to overlay assets if underlying assets are mishandled; interfaces with existing institutions and systems; user identification across systems; risks posed by sharing blockchain security through merged mining or anchoring}	
%	\item{Susceptibility to coordinated attacks by large parties}
%	\item{Off-chain functionality: off-chain program execution; interoperability with off-chain systems}
%	\item{Privacy: anonymity; confidentiality}
%	\item{Resilience: distributed denial-of-service (DDOS) attacks; dishonest majority attacks; security of infrastructure; subversion of software security measures}		
%	\item{Scalability: block generation frequency; block size limits; computational cost of public blockchains; scalable and secure end-user software}
%	\item{Security}
%	\item{Smart contract correctness: inherent incompleteness of contracts; ensuring completeness; lack of tools for verification}
%\end{itemize}
%
%\subsubsection{Socio-Technical Challenges}
%\begin{itemize}
%	\item{Blockchain is an inefficient use of computing resources}
%	\item{Cryptocurrency economics: illiquidity, price volatility, high initial adoption costs, currency conversion costs, and decreasing marginal returns for miners}
%	\item{Incentives: correctly configuring game-theoretic incentive structures}
%	\item{Key management: difficulty of manual key management; possible unrecoverable loss of private keys; attacks against wallets; bugs and glitches in wallets}	
%	\item{Lack of protection against mistakes: transactions cannot be reversed; administrators cannot restore access if users are locked out}
%	\item{Miner centralization}
%	\item{Usability: difficulty of developing distributed apps; inherent complexity of technology; difficulty of access and use by consumers; education; onboarding users; difficulty of search; poor UX; lack of mobile and web clients}
%\end{itemize}
%
%\subsubsection{Challenges to Market Viability}
%\begin{itemize}
%	\item{Efficiency and cost: low or bottlenecked throughput; wasteful energy consumption; high transaction fees; latency induced by synchronous communication required by certain consensus protocols}
%	\item{Expense of developing end-user applications for individual blockchains}
%	\item{Usefulness: few demonstrable use cases; does not improve upon existing solutions in many domains being pursued}
%	\item{Unnecessary: if a central party is required; if a trusted intermediary exists; if a small number of parties are involved in the system}
%\end{itemize}
%
%\subsubsection{Use Case Challenges}
%\begin{itemize}
%	\item{Binding digital entities to real-world entities: stapling tokens to assets; interoperating with existing systems (e.g., compatibility between cryptocurrencies and cash)}
%	\item{Dispute resolution: difficulty recovering from errors or bugs; difficulty reversing fraudulent transactions; non-applicability to scenarios where a central broker is needed}
%\end{itemize}
%
%\subsubsection{Regulatory Challenges}
%\begin{itemize}
%	\item{Governance: resolution of conflicts requiring external intervention; incident response; rule updates require forks; transparency of software development}
%	\item{No distinct legal framework}
%	\item{Standardization}
%	\item{Regulatory: anti-money laundering, know-your-customer; difficulty of monitoring; lack of regulation; legal considerations; taxation; exchange control and flow management; consumer protection}	
%	\item{Cryptocurrencies are inherently Ponzi schemes}
%	\item{Reputation: use for crime; associations with black markets; nebulous or illicit uses; terrorist financing}
%\end{itemize}

% Jeremy: Arkady -- please see reclustering of previous section and use it to motivate why these are the right subset to study.
% Is this subset of challenges normative or positive?
% Normative -- the set of challenges that ought to be studied
% Positive -- the set of challenges that happened to be studied by others

Our analysis has allowed us to identify several challenge areas for the development and application of blockchain technology.  In this section we discuss several of these challenges and briefly summarize related approaches for mitigating them.

%\paragraph{Scalability of public blockchains: transaction costs and bandwidth, emergent centralization, and energy consumption}
\subsection{Blockchain scalability}
As blockchain-based systems such as Bitcoin have grown, the increased number of transactions and users have raised questions about the scalability of such systems.  In particular, today's systems often waste large amounts of energy, have slow transactions with high transaction fees, and have trouble supporting high transaction volume.

\subsubsection{Power consumption and centralization of mining}
Most of today's permissionless blockchains are based on proof-of-work style consensus in which consensus is achieved by having miners solve computationally difficult puzzles to extend the blockchain.  This has led to the proliferation of specialized hardware devices dedicated to solving these computational puzzles.  However, such devices consume tremendous amounts of power in order to maintain these blockchains.  For example, according to one estimate\cite{Digiconomist}, as of April 2018 the energy consumed by Bitcoin miners was equivalent to the power usage of almost 5.5 million US households, and all signs indicate that it will continue to grow.

To reduce the power consumption of Blockchain, there have been several proposals to turn to other consensus mechanisms that do not rely on proofs of work.  The most popular of these proposals is proof of stake consensus, where parties' contribution to the consensus protocol is proportional to the total amount of stake they own in the system rather than the amount of work that they do.  This allows consensus to be achieved without relying on wasteful proofs of work.  Today's proof of stake protocols (e.g.~\cite{FC:BenGabMiz16,eprint:BenPasShi16,CRYPTO:KRDO17,SOSP:GHMVZ17}) vary significantly in their model, assumptions, and performance guarantees.  Other suggestions for avoiding proofs of work include proof of space~\cite{CRYPTO:DFKP15, eprint:PPKAFG15} where miners use storage instead of computation, and proof of elapsed time~\cite{SSS:CXSGLS17} where trusted hardware (i.e., Intel SGX) is used in place of proofs of work.  It is not clear at this point which solution will be best suited for different blockchain deployments.  

Another unintended consequence of proof-of-work blockchains is the centralization of mining power.  In order to reduce variance in their earnings, miners are incentivized to work together in large mining pools, pooling their computing power and sharing the profits among pool members.  Currently, almost 70\% of Bitcoin blocks are mined by the five largest mining pools \cite{BlockchainInfoPools} significantly limiting the actual decentralization achieved by Bitcoin~\cite{arxiv:GBERS18}. Of course, these pools are disincentivized to damage trust in Bitcoin (and thus reduce its value and their profits) by abusing their power to censor transactions or violate rules in other ways. But this centralization undoubtedly runs counter to the normative property that Blockchain is decentralized and may violate security notions that depend on decentralization.  One possible solution~\cite{CCS:MKKS15} is to discourage mining pool formation by making it impossible to enforce cooperation between the members.

\subsubsection{Increasing transaction rates}
Another challenge to the scalability of Blockchain solutions, especially in the permissionless setting, is the increasing number of transactions.
Current systems often have rather long wait times before a transaction can be confirmed on the blockchain (e.g., Bitcoin can take several hours to confirm a transaction~\cite{BlockchainInfoTransactionConfTime}).  This makes blockchain-based solutions less than ideal when immediate transactions are needed, such as when purchasing physical goods.

A couple of different approaches have been proposed to deal with this issue.  First, a number of hybrid consensus algorithms (e.g.,~\cite{SOSP:GHMVZ17,OPODIS:AMNRS17,DISC:PasShi17,EC:PasShi18,NSDI:EGSR16}) aim to reduce transaction approval times through reducing or eliminating forks. Most of these work by using ``proof-of'' style protocols to elect a committee or a leader who then uses traditional byzantine fault-tolerant consensus to advance the blockchain.  A second approach for improving transaction rates, especially for financial transactions, is to make use of payment-channel networks. Such networks set up pairwise channels between parties to allow transactions on these channels to occur ``off-chain'', i.e., without being recorded on the blockchain; the blockchain is only used for conflict resolution.  Many different flavors of payment-channel networks achieving various properties have been proposed (e.g.~\cite{PooDry16, NDSS:HABSG17,CCS:KhaGer17,SYSTOR:LNEKPS18,CCS:MMKMR17,CCS:GreMie17}) and several such as the Lightning network~\cite{PooDry16} are in active development for financial transactions on top of Bitcoin.  

\subsubsection{Handling increased transaction volume}
Another scalability challenge for popular blockchain-based systems such as Bitcoin and Ethereum is the sheer volume of transactions that are being added to the blockchain.  As more and more third-party services start to use these blockchains to store and execute their transactions, these systems have to verify and store transactions for a variety of unrelated operations. This can cause the storage and verification work required of miners to become prohibitively expensive.

Proposed solutions for this problem include sharding (e.g.~\cite{CCS:LNZBGS16, FC:GenRenSir17}) to partition transactions based on the transaction type or service.
This allows different sets of miners to verify transactions for different services thus reducing the amount of verification work each miner must do.  Another more radical approach to deal with this challenge has been to move away from the ``chain'' view of blockchain.  Instead, several proposals (e.g., ~\cite{ePrint:SomLewZoh16,eprint:SomZoh18,IOTA}) propose to organize transactions into a directed-acyclic graph (DAG) where later transactions can vote on the validity of earlier transactions, allowing transactions to be approved before global consensus is achieved.
%As the Bitcoin user base and transaction volume have grown, limitations on its scalability have been revealed. To some degree, these limitations are a result of Bitcoin's chosen parameters; for example, the 1MB cap on block size has periodically caused prohibitively high transaction fees and wait times. This limitation was mitigated by the Bitcoin Cash hard fork in 2017, which created 8MB blocks, but other scalability problems may not be so easily solvable by simple parameter tweaking, suggesting that they may be inherent limitations of public blockchain technology. We will discuss two limitations that stem from the proof-of-work consensus model that is typically used in public blockchains: emergent centralization of control and high rates of energy consumption.
%
%Currently, almost 70\% of Bitcoin blocks are mined by the five largest pools \cite{BlockchainInfoPools}. Of course, these pools are disincentivized to damage trust in Bitcoin (and thus reduce its value and their profits) by abusing their power to censor transactions or violate rules in other ways. But this centralization undoubtedly runs counter to the normative property that Blockchain is decentralized and may violate security notions that depend on decentralization. This problem is caused in part by the hardware-optimized Bitcoin mining puzzle, for which alternatives are being explored, but whether or not this limitation can be overcome at scale is still an open question.
%
%Bitcoin is also an illustrative example for another inefficiency of public blockchains at scale: the energy consumption of proof-of-work consensus. According to one estimate\cite{Digiconomist}, as of April 2018 the energy consumed by Bitcoin miners is equivalent to the power usage of almost 5.5 million US households, and all signs indicate that it will continue to grow. A similar problem is likely to arise in any scheme where proof-of-work is deployed. It may be surpassable if alternative consensus protocols such as proof-of-stake are used, but there are other problems to work out with such schemes (in particular, a worsening of the emergent centralization problem discussed above). 


%\noindent
%\textbf{Academic refs:}\\
%Proof of stake:
%\begin{enumerate}
%\item ~\cite{CRYPTO:KRDO17} - provably secure proof of stake
%\item ~\cite{eprint:BenPasShi16} - provably secure proof of stake
%\item ~\cite{SOSP:GHMVZ17} - ALGORAND - provably secure proof of stake.  Works in both permissioned and permissionless setting
%\item ~\cite{FC:BenGabMiz16} - original academic PoS proposal?
%\item ~\cite{DISC:PasShi17} - Show loss of responsive is necessary if secure vs. $t\ge n/3$.  Show hybrid consensus for $t< n/3$ uses PoW and byzantine consensus to select committee.
%\item ~\cite{AC:PasShi17} - Consensus in sleepy setting with weakly synchronized clocks (for $t<n/2$)
%\item ~\cite{CRYPTO:DFKP15} - proof of space
%\item ~\cite{eprint:PPKAFG15} - blockchain based on proof of space
%\item ~\cite{OPODIS:AMNRS17} - Improves bitcoin confirmation time, based on reconfigurable byzantine consensus augmented with PoW.
%\item ~\cite{eprint:FanZho17} - aim to mimic Nakamoto design via PoS
%\end{enumerate}
%Payment-channel networks:
%\begin{enumerate}
%\item ~\cite{NDSS:HABSG17} - fast anonymous payments using (untrusted) tumbler, off-chain
%\item ~\cite{PooDry16} - original Bitcoin lightning network white paper - fast off-chain payments
%\item ~\cite{SYSTOR:LNEKPS18} - payment networks using TEEs.  Avoid some requirements of prior payment networks.
%\item ~\cite{CCS:KhaGer17} - formal treatment of rebalancing payment channels, to allow refilling of channel (by moving funds from other channels) without on-chain transactions.  On-chain transactions only in case of dispute.
%\item ~\cite{CCS:MMKMR17} - private, UC-secure payment neworks
%\item ~\cite{CCS:GreMie17} - anonymous payment channels
%\end{enumerate}
%Sharding:
%\begin{enumerate}
%\item ~\cite{CCS:LNZBGS16} - propose ELASTICO - splits transactions into shards, each shard verified by different miners
%\item ~\cite{FC:GenRenSir17} - propose sharding as solution (Aspen) for multi-service blockchains.  Break transactions into blocks based on service they are for.
%\item ~\cite{NSDI:EGSR16} - Bitcoin-NG (next generation) - idea - elect leader, and have him do all serialization of transactions for an epoch
%\end{enumerate}
%IOTA-style:
%\begin{enumerate}
%\item~\cite{EC:PasShi18} - PoW-based, tolerates $t<n/2$, but if $t<n/4$ then can get transaction confirmed at network rate.
%\item ~\cite{FC:SomZoh15} - study security of various fast block-creation rates for Bitcoin.  Also, propose GHOST rule for choosing which chain to keep based on heaviest subtree rooted at branch.
%\item ~\cite{ePrint:SomLewZoh16} - SPECTRE - accept transactions in seconds, DAG-based, blocks in DAG are voters, vote on order of blocks in DAG
%\item ~\cite{eprint:SomZoh18} - PHANTOM blockDAG protocols, blocks vote on which other blocks were mined honestly.  Generates robust full order of blocks that is eventually agreed upon by all nodes.
%\end{enumerate}

%\paragraph{Smart contract correctness}
\subsection{Smart contract correctness and dispute resolution}
All executable code is subject to bugs: developer errors that can be taken advantage of to hijack program logic. This problem manifests in smart contracts, and when those contracts control the transference of valuable assets, the impact of a bug can be devastating. The immutability of blockchains exacerbates this challenge by impeding rollback of state changes, even those that are clearly malicious. This is because, by definition, any transactions on a blockchain upon which consensus is reached are considered legal - including ones due to buggy code and exploitations of such. If ``code is law'', as claimed by a blockchain-based investment fund called the Decentralized Autonomous Organization (DAO), then so are bugs. 

\subsubsection{Eliminating bugs in smart contracts}
This principle of ``code is law'' was put to the test in June 2016 when a bug in the contract allowed an attacker to drain the DAO of about \$80m in tokens. 
%Targeting a smart contract that first moved funds and then updated account balance, the attacker exploited a vulnerability to recursively execute the contract in between those two steps. %Repeatedly moving funds without ever checking account balances allowed the attacker to move more funds than were authorized. 
In response, the DAO developers implemented a "hard fork" from a state before the attack occurred. This created two blockchains: one in which the funds had never been drained and another in which the attacker still held their spoils \cite{Castillo16}. Of course, this rollback contradicts the perception that the blockchain is immutable and settled contracts are final; we will discuss this contradiction later as a key limitation of Blockchain technology. The challenge is how to ensure smart contract correctness to eliminate such attacks and to build in recovery mechanisms in case that fails.

Three different directions have been proposed for improving the correctness and security of smart contracts:  Education and tools to help developers write smart contracts, tools for evaluating correctness and security of existing smart contracts, and formal modeling and formal verification of smart contracts. Along the education path, researchers organized a class on developing smart contracts cataloging common mistakes and misunderstandings~\cite{FC:DAKMS16}. Additionally, tools have been developed to simplify development of \emph{private} smart contracts~\cite{SP:KMSWP16}. For evaluation of existing smart contracts, multiple tools using symbolic execution~\cite{CCS:LCOSH16}, machine learning~\cite{arxiv:Huang18}, and static analysis~\cite{CCS:BDFGGK+16,NDSS:KGDS18} have been developed for detecting bugs and vulnerabilities. Finally, some efforts to support development of formally verified smart contracts is underway; for example, the Ethereum Virtual Machine (EVM) has been fully defined for interactive theorem provers~\cite{Hirai17}, which are essential tools for building formally verified software of any kind. Magazzeni et al.~\cite{Magazzeni17} have laid out a research agenda identifying further groundwork that must be conducted to support formal verification of smart contracts.


%\noindent
%\textbf{Academic refs:}
%\begin{enumerate}
%\item~\cite{CCS:LCOSH16} - build Oyente -- a symbolic execution tool to help detect potential security bugs.  Found 8833 of 19.366 existing Ethereum smart contracts are vulnerable
%\item~\cite{SP:KMSWP16} - enable private smart contracts (i.e., ones that dont post transaction data in the clear.  Also, have crypto model of smart contracts programming, and automatically generate smart contract and code for users to interact with smart contract -- not sure this is relevant here.
%\item~\cite{FC:DAKMS16} - document experience of teaching a class on writing secure smart contracts.  Suggest best practices and open course material.
%\item~\cite{arxiv:Huang18} - turn bytecode of Solidity into RGB map and then use convolutional neural nets to detect bugs - maybe drop?
%\item~\cite{CCS:BDFGGK+16} - MSR+Harvard folks - F* platform to statically evaluate security of smart contracts.
%\item~\cite{NDSS:KGDS18} - IBM Research - framework to validate correctness and fairness of smart contracts.  Found 94.6\% of smart-contracts are vulnerable.
%\end{enumerate}

\subsubsection{Dispute resolution and rollback of mistakes}
Despite best efforts to eliminate mistakes in smart contract and transactions, a payment or asset transfer system must be able to reverse fraudulent or errant transactions. The immutability property of Blockchain means that transactions cannot be stricken from the ledger after consensus has been reached, so alternative means of dispute resolution must be explored. A new transaction reversing the effects of the disputed transaction could be added to the ledger, but decentralized governance makes arbitrating such a dispute difficult as there is no individual arbiter with the authority to determine which party is in the right when a dispute occurs.  Additionally, dispute resolution must be handled carefully to avoid introducing new vulnerabilities.  For example, several attacks were demonstrated against the Bitcoin refund mechanism~\cite{FC:MccShaHao16} necessitating further research to design secure refunds in Bitcoin~\cite{arxiv:AviSafSha18}.

%\noindent
%\textbf{Academic refs:}
%\begin{enumerate}
%\item ~\cite{arxiv:AviSafSha18}- design new protocol for security against Bitcoin refund attacks.
%\item ~\cite{FC:MccShaHao16} - original paper showing refund attacks on Bitcoin
%\end{enumerate}

\subsection{Stapling between on-chain transactions and real-world objects}
Another challenge that arises in many Blockchain applications is that of establishing correspondence between on-chain digital tokens and off-chain goods that those tokens represent as well as off-chain information that can serve as input in transactions.  

\subsubsection{Consistency of off-chain assets and corresponding on-chain tokens}
When a blockchain is used to track off-chain assets (physical or digital), those assets are typically represented on-chain by digital tokens. When dealing with digital assets, correspondence between the asset and its token can generally be maintained by code; for example, a smart contract can track transference of ownership for a digital media license. For physical assets, however, maintaining this consistency is a challenge. Real-world processes must be employed to ensure that whenever an asset's state or ownership is modified, the corresponding token is updated. These processes are an obvious point of failure as they rely on correct enforcement by trusted parties. The person in charge of maintaining the blockchain could attach two tokens to one asset, two assets to one token, or issue tokens that have no backing asset (e.g. stocks in a naked short selling scenario). The end user must also be trusted, as they may be able to separate the token and sell it while keeping the asset, freeing the token to be attached to an invalid asset (e.g., fake goods in luxury markets). 

%\noindent
%\textbf{Academic refs:}
%\begin{enumerate}
%\item~\cite{FC:GBGN17}- present secure escrow services that can enable purchase of physical good using Bitocin - not sure this is relevant
%\end{enumerate}



\subsubsection{Non-auditability of off-chain oracles}
Smart contracts sometimes leverage off-chain oracles: online services that provide information in response to a request. For example, gambling contracts may determine which address to pay winnings to based on the result of some oracle request (e.g., sports scores, stock prices, weather forecasts, or other global events). If contract logic branches based on the response from an off-chain oracle, that contract is no longer verifiable after the fact because auditors cannot confirm that the response received from the oracle at audit time is the same response received when the contract was executed. There are legitimate reasons why an oracle response might change with time, so this is really an inherent limitation of Blockchain technology: smart contracts cannot "see" external events.

One system tackling this problem is Town Crier~\cite{Zhang16}. It uses a trusted hardware back end (Intel SGX) to serve as an authenticated off-chain oracle for Ethereum smart contracts. This allows an auditor to later verify the oracle's response at the time at which the contract was executed. This solution sidesteps the limitation of oracle non-auditability by moving trust to a centralized location (the hardware platform providing authenticity), but it does not diffuse trust in the way our analysis reveals that Blockchain technology is expected to.

\subsection{Key Management}
Another major challenge of blockchain is that it requires users to store, manage, and secure cryptographic keys both to certify their own transactions and to verify transactions of others.  However, the challenges of maintaining and protecting such key repositories are well documented (see, e.g.~\cite{uss:WhiTyg99}).  A survey by Eskandari et al.~\cite{arxiv:ECBS18} outlines the challenges as well as potential solutions for managing keys for Bitcoin.  They discuss solutions such as password-protected and password-derived keys as well as offline and air-gapped storage of the keys.  But, as the authors state all of these solutions have their drawbacks.

Some recent work has also looked into cryptographic solutions to protect users' keys.  Techniques based on secure multi-party computation (MPC)~\cite{CCS:LinNof18,C:Lindell17} allow transactions to be signed without any party ever having access to the secret key.  Alternatively, the classic technique of threshold signatures (e.g.~\cite{PKC:Boldyreva03,EC:GJKR96,EC:Shoup00a}) allow users to split their keys into many pieces such that a large number of them must be compromised in order to ``steal'' a user's key.  However, much work remains to secure all the cryptographic keys inherent in real-world blockchain deployments.

\subsection{Regulation}
As applications of blockchain technology proliferate, they have drawn significant attention from the regulatory bodies around the world.  In the settings of cryptocurrencies, a number of concerns such as the prevalence of black-market transactions, tax evasion, money laundering, and terrorist financing have drawn calls to regulate how such cryptocurrencies can be used.  An excellent review article by Kiviat~\cite{Kiviat15} outlines some of the issues that arise in regulating blockchain transactions.  Additionally, as blockchain applications move to greater support of smart contracts, researchers have shown that criminal smart contracts are easily implementable in today's smart contract platforms~\cite{CCS:JueKosShi16} requiring regulation to avoid such criminal uses of blockchain.  

Another regulatory issue that has recently come to light is whether blockchain technology is compatible with European Union's General Data Protection Regulation (GDPR) that has recently been put into law in the EU.  GDPR requires that individuals have the ``right to be forgotten'', i.e., that they should be able to expunge outdated or invalid information about themselves from available records.  However, since any data stored on a blockchain is immutable, this seems to pose a challenge for GDPR compliance.  Much research within the technical and regulatory communities is needed to ensure that blockchain can comply with existing regulation and to prevent the many nefarious uses of this technology.

