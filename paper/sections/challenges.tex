% !TEX root = ../main.tex

%TODO: Add back in if this makes it into the main body of the paper. Even in the appendix this could be added back.
%\subsection{Academic Literature Review}
%As part of the grounded theory analysis, the data revealed several open research challenges related to Blockchain technology.
%In regards to these challenges, we conducted a review of academic literature to identify what research has already been done and what the academic community thinks of these challenges.
%These challenges, along with the relevant paper references, are discussed in Section~\ref{sec:research-challenges}.

\section{Challenges for Blockchain Technology}
\label{sec:challenges}

In this section, we describe challenges and limitations that must be addressed when building applications for Blockchain technology.

%\paragraph{Scalability of public blockchains: transaction costs and bandwidth, emergent centralization, and energy consumption}
\subsection{Blockchain scalability and performance}
As usage of Blockchain applications such as Bitcoin has grown, questions have been raised about their scalability, cost, and performance. Many documents in our corpus point out that blockchain is \emph{not a high performance system} and that its decentralized governance and operation result in an inefficient use of resources.  In particular, most of today's permissionless blockchains are based on proof-of-work style consensus in which consensus is achieved by having miners solve computationally difficult puzzles. This has led to the proliferation of power-hungry specialized hardware devices dedicated to solving these computational puzzles. An estimate in April 2018 found that the energy consumed by Bitcoin miners was equivalent to the power usage of almost 5.5 million US households~\cite{Digiconomist}, and all signs indicate that it will continue to grow.

Another unintended consequence of proof-of-work blockchains is the centralization of mining power.  In order to reduce variance in their earnings, miners are incentivized to work together in large mining pools, pooling their computing power and sharing the profits among pool members.  Currently almost 70\% of Bitcoin blocks are mined by the five largest mining pools~\cite{BlockchainInfoPools}, inhibiting decentralization~\cite{arxiv:GBERS18}. Of course, these pools are disincentivized to damage trust in Bitcoin (and thus reduce its value and their profits) by abusing their power to censor transactions or violate rules in other ways. But this centralization undoubtedly runs counter to the normative property that Blockchain is decentralized and it may violate security notions that depend on decentralization.  

Additionally, many design choices limit the performance of Blockchain systems.  Small block sizes can lead to extremely high transaction fees when miners are unwilling to add transactions that result in having to create more blocks.  On the other hand, larger block sizes can impede  throughput due to the increased latency of finalizing a transaction. In Bitcoin, one must wait for six additional blocks to be confirmed before a transaction is considered final, which may be prohibitively long for applications where transactions need to clear immediately. The challenge of handling a huge number of transactions while achieving acceptable levels of throughput and latency remains a critical hurdle to the use of Blockchain solutions in performance-critical systems.

\subsection{Blockchain correctness and dealing with errors}
A payment or asset transfer system must be able to reverse fraudulent or errant transactions. The immutability property of Blockchain means that transactions cannot be stricken from the ledger after consensus has been reached, so alternative means of dispute resolution must be explored. A new transaction reversing the effects of the disputed transaction could be added to the ledger, but decentralized governance makes arbitrating such a dispute difficult as there is no individual arbiter with the authority to determine which party is in the right when a dispute occurs.  Additionally, dispute resolution must be handled carefully to avoid introducing new vulnerabilities.  For example, several attacks were demonstrated against the Bitcoin refund mechanism~\cite{FC:MccShaHao16} necessitating further research to design secure refunds~\cite{arxiv:AviSafSha18}.

Correctness is also a challenge in smart contracts. The impact of a bug in a contract can be devastating what that contract controls the transference of valuable assets, and the immutability of Blockchain exacerbates this challenge by impeding rollback of state changes. This is because by definition, any transactions on a blockchain upon which consensus is reached are considered legal - including those due to buggy code and exploitations of such. If ``code is law'', as claimed by a blockchain-based investment fund called the Decentralized Autonomous Organization (DAO), then so are bugs.  Moreover, the lack of tools for checking correctness of smart contracts before they are deployed significantly increases the likelihood of such bugs.  The risk of such bugs was made plainly clear in June 2016, when a bug in the DAO smart contract allowed an attacker to steal about \$80m.

A final challenge to the correctness of Blockchain application is ensuring consistency between on-chain state and the off-chain objects or state that it represents.  When a blockchain is used to track off-chain assets (physical or digital), those assets are typically represented on-chain by digital tokens. When dealing with digital assets, correspondence between the asset and its token can generally be maintained by code; for example, a smart contract can track transference of ownership for a digital media license. For physical assets, however, maintaining this consistency is a challenge. Real-world processes must be employed to ensure that whenever an asset's state or ownership is modified, the corresponding token is updated. These processes are an obvious point of failure as they rely on correct enforcement by trusted parties. The person in charge of maintaining the blockchain could attach two tokens to one asset, two assets to one token, or issue tokens that have no backing asset (e.g. stocks in a naked short selling scenario). The end user must also be trusted, as they may be able to separate the token and sell it while keeping the asset, freeing the token to be attached to an invalid asset (e.g., fake goods in luxury markets).

Another related issue arises when Blockchain processes and smart contracts need to use off-chain inputs.  For example, gambling contracts may determine which address to pay winnings to based on the result of a request to an off-chain oracle (e.g., sports scores, stock prices, weather forecasts, or other global events). If contract logic branches based on that response, the contract is no longer verifiable because auditors cannot confirm that the response received from the oracle at audit time is the same response received when the contract was executed. There are legitimate reasons why an oracle response might change with time, so this is really an inherent limitation of Blockchain technology: smart contracts cannot "see" external events.  Thus, additional mechanisms are needed to verify that these inputs cannot be forged.

\subsection{Security issues of Blockchain}
Due to its decentralized nature, Blockchain is potentially vulnerable to a number of security threats.  First, coordinated attacks by a majority (or often even a large minority) of the miners can reorder, remove, and change transactions from the blockchain.Thus, it is critical that Blockchain applications provide the proper incentives to keep miners honest.  In particular, it is critical to design incentives such that the Nash equlibrium behavior for selfish miners is to honestly maintain the state of the blockchain rather than to destroy it through forming such coalitions.

Additionally, Blockchain applications are potentially vulnerable to traditional network attacks such as denial of service or partitioning.  Such attacks can aim to bring down a fraction of the mining power or to fracture the network of miners to create an inconsistent state.  Thus, Blockchain applications must be able to recover to a valid state even after such a denial of service attack.
\anote{Not sure what security of infrastructure -- is this key management?, and subversion of software security measures means.}

\subsection{Privacy and anonymity}
Another major challenge is how to protect the privacy of the users and data stored on a blockchain.  By design, all data on a blockchain is public to enable verification by all miners.  But this means that any sensitive data is inherently non-private.  Cryptographic approaches such as encryption and zero-knowledge proofs are needed\anote{Add citations} to protect the privacy of such data.

Moreover, if one wishes to hide the parties participating in a transaction recorded on a blockchain, it is not enough to hide the transaction information.  It is also necessary to protect the anonymity of these parties by hiding the fact that they connected to the blockchain in the first place.  While most blockchains provide a notion of ``pseudonymity'' in which users are identified by their cryptographic keys instead of by their names or social security numbers, it has been shown that this is not enough to provide true anonymity as attacks that correlate transactions by the same pseudonyms together with access histories can deanonymize blockchain transactions\anote{Add citations}.

\subsection{User Experience}

The requirement for blockchain users to store, manage, and secure cryptographic keys is a major challenge. Key management is needed to certify users' own transactions, as well as to verify the transactions of others.  The challenges of maintaining and protecting key repositories are well documented in other areas of cryptography (see, e.g.~\cite{uss:WhiTyg99} for secure email). A survey by Eskandari et al.~\cite{arxiv:ECBS18} outlines the challenges as well as potential solutions for managing keys for Bitcoin. They discuss solutions such as password-protected and password-derived keys as well as offline and air-gapped storage of the keys.  But as the authors state, all of these solutions have their drawbacks.

With a key management solution in place, a further challenge is obtaining transactions for auditing purposes---auditing is a much touted capability of blockchain across our dataset. In a decentralized environment, running a full node on the network, or relying on a entity that does (such as a blockchain explorer website or a lightweight user-client) is necessary. 

% Jeremy: I toned this waaaaaay down. Ethereum has excellent tools now: linters, debuggers, static analysis tools. Even the etheruem wallet can compile and do basic debugging of Solidity code. However, beyond Ethereum things get messy and so that is the new point. 
For developers, development and analysis tools are critical to building secure applications in any domain, including blockchain. The availability of user-friendly developer tools varies significantly depending on the maturity of the target blockchain. Some projects, like Ethereum have mature tools, while others have very little support. Many blockchains are currently geared towards expert users and lack the user experience-focused tools needed to allow for easier use by non-experts.

 
%Yet another important challenge that arises when building applications on top of Blockchain is the lack of development and analysis tools.  %Currently there are no universal tools available for creating transactions and smart contracts for many of the existing blockchains, this is even more true if we want tools that can be used across multiple blockchains. - \bnote{I'm not sure this is true -- there are IDEs emerging for some smart contract languages now, like Solidity.}
%As mentioned earlier, the lack of smart contract development and testing tools increases the chance of critical bugs.  Additionally, there is a lack of testing support and infrastructure to test out one's applications (for correctness, performance, and security) before deploying them. 

\subsection{Key management challenges}

\subsection{Necessity of blockchain}
Finally, before a company chooses to deploy Blockchain it should carefully consider whether it is the right tool for the job.  Due to the hype around this technology many solutions have turned to Blockchain when other, more established technologies such as traditional databases would do the job. For example, in situations when a centralized repository is easily available, it is likely to be preferable to a blockchain.  In fact there are relatively few use cases where blockchain is demonstratively the ``right'' solution.  While we believe that more such use cases will be identified, when evaluating a new use case it is important to make sure that Blockchain is really needed and that a lower-overhead solution will not suffice.


\subsection{Legality and Regulation}

Our analysis revealed wide-spread concern with regulatory issues surrounding cryptocurrencies, blockchain-based assets, and other blockchain applications. It is important to note that regulation applies indirectly to technology, based on how the technology is used within a firm's operations. Therefore there will be no direct regulation of Bitcoin, for example, but rather regulation of firms that use Bitcoin according to how they are using it. Consider the example of a Bitcoin exchange service that derives income from fees paid in Bitcoin: it will have to consider tax declarations as a business (\eg service taxes and capital gains), financial reporting as a money service business (\eg know your customer, anti-money laundering and anti-terrorist financing), generally acceptable accounting standards for audited financial reports (\eg reporting Bitcoin as an intangible asset on a balance sheet), and potentially additional regulation that applies to financial exchanges, banks, and/or custodians. In most countries, each of these already broad categories are administered by different government branches. Countries like the United States and Canada may require licensing or registration, and have taken enforcement action against non-compliant firms.

An extreme case of regulation is prohibition of cryptocurrencies or blockchain-assets. At the time of writing, the largest country to ban Bitcoin is Pakistan and the largest country to prohibit wide categories of cryptocurrency use is China.
 
 
%As applications of blockchain technology proliferate, they have drawn significant attention from the regulatory bodies around the world.  In the settings of cryptocurrencies, a number of concerns such as the prevalence of black-market transactions, tax evasion, money laundering, and terrorist financing have drawn calls to regulate how such cryptocurrencies can be used.  An excellent review article by Kiviat~\cite{Kiviat15} outlines some of the issues that arise in regulating blockchain transactions.  

% Jeremy: I like this paper but it is a bit esoteric to real world concerns:
%Additionally, as blockchain applications move to greater support of smart contracts, researchers have shown that criminal smart contracts are easily implementable in today's smart contract platforms~\cite{CCS:JueKosShi16} requiring regulation to avoid such criminal uses of blockchain.  

% Jeremy: This is too speculative in my opinion and there is enough to talk about otherwise:
%Another regulatory issue that has recently come to light is whether blockchain technology is compatible with European Union's General Data Protection Regulation (GDPR) that has recently been put into law in the EU.  GDPR requires that individuals have the ``right to be forgotten'', i.e., that they should be able to expunge outdated or invalid information about themselves from available records.  However, since any data stored on a blockchain is immutable, this seems to pose a challenge for GDPR compliance.  Much research within the technical and regulatory communities is needed to ensure that blockchain can comply with existing regulation and to prevent the many nefarious uses of this technology.

