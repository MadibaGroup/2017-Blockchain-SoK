% !TEX root = ../main.tex

\section{Discussion (All)}

Fast/cheap is because of deregulation
	Do things that are legally questionable

Degrees of decentralization
	Single - not really a blockchain, or is it?
	Oligarchy
	Embarrassingly decentralized
	
There are risks of thinking they are the same
	Using blockchain assuming you get normative properties, not just technical properties
	Get saturated on normative properties, ignore the technical properties
	
Blockchain is not only for global system

Lighter options when all of blockchain not needed
	Distributed data sources with auditability/replication
	
\paragraph{Ideology, hype, and ulterior motives}
Many proponents of Blockchain believe that it has the capability to massively disrupt how society operates, or at least to rapidly overtake legacy solutions in many significant industries. This belief is hyperbole ("hype") because although it has been validated under certain conditions, it has not been demonstrated to be generally true but it is still fervently proclaimed. This ideology and hype cause problems: for example, frequent emotionally-charged schisms within Blockchain advocate and developer communities - especially those affiliated with Bitcoin. This turmoil prevents level-headed scientific discourse and wastes developer resources. It can also tangibly affect the stability of a blockchain-based system by causing a fork, in which two independent chains emerge to used and maintained by different groups, further dividing resources.

With that said, Blockchain's disruptive power has certainly been demonstrated in the financial sector, so it clearly has promise. Several factors have made this sector an attractive target for disruption, perhaps none more so than the opportunity for massive profit. This motive has had benefits for Blockchain technology, especially in accelerating the pace of technological development. However, it has also created perverse incentives to reinforce hype and ideology. Hype can attract investors and inflate valuations, and dogmatic ideology is a proven marketing and recruitment strategy for financial scammers. These problems inhibit the advancement of Blockchain technology.

\paragraph{Reputation for illicit uses}
Due to the prominence of Bitcoin, many people are familiar with Blockchain first and foremost as the technology underlying the cryptocurrency and therefore the reputations of the two are intertwined. The fact that Bitcoin is designed to avoid banks and central authorities in general, combined with its well-known history of illicit uses, somewhat poisons the well for Blockchain as a whole. Along with the causes listed above (ideology, hype, and ulterior motives), this contributes to the difficulty of discussing and considering Blockchain technology with precision and objectivity. It may also have impeded or delayed its acceptance by organizations unwilling to associate themselves with the technology's poor reputation.

\paragraph{Inefficiency of replicated storage}
Typically, data on a blockchain is stored redundantly by all peers maintaining the chain. For some use cases, and particularly when a blockchain is being used as a distributed data store, this results in prohibitively large data storage requirements for those peers. In Bitcoin, for example, a user wishing to verify transaction integrity back to the first block must store over 160GB of data\cite{BlockchainInfoSize} with approximately an additional 4GB added each month\footnote{For Bitcoin, solutions have been explored that would reduce storage requirements (e.g., Simple Payment Verification\cite{NakamotoS8}) and blockchain growth rate (e.g., Lightning network\cite{Poon16}). However, the inherent limitation that any truly verifiable data must reside on the blockchain in some form and be stored by all peers remains.}. This limitation inhibits blockchain-based solutions to problems with large data storage requirements, such as medical record storage.