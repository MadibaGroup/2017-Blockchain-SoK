% !TEX root = ../main.tex

\section{Application}

Ultimately, blockchain technology is not a panacea but it is a useful tool when the overhead is justified by the system's needs. A good place to start is the follow questions. 

% %Before describing how shared governance and operation defines Blockchain technology, it is important to define both of these concepts.
%Governance is the process by which one or more entities (e.g., an individual, a company, a government agency) define how a system will operate---i.e., what operations are supported and how those operations function.
%Governance can be singular---a single entity (e.g., a developer, a company) defines the system---or it can be shared---multiple entities work together to define the system (e.g., IETF, RFC).
%Operation refers to the number of parties that actually operate a system.
%Operation can be singular---a single entity executes the system (e.g., a trusted third party)---or shared---multiple entities collectively operate the system (e.g., through the use of a consensus protocol).
%Note, distributed operation (i.e., operation on many machines) does not imply shared operation, as the owner/operator of all distributed nodes may still be a single entity.

%Based on our results and experience, we recommend the use of the following 
%questions to determine if Blockchain technology would be a good fit for a 
%specific project.

\begin{enumerate}
	\item Does the system require shared governance?
	\item Does the system require shared operation?
\end{enumerate}

Governance is the process by which one or more entities (e.g., an individual, a company, a government agency) define how a system will operate and operation is the day-to-day deployment of the system. If both answers are no, the overhead of a blockchain is unnecessary. If both answers are yes, there is a good fit. If only one of shared governance or shared operation is needed, then the following two questions can be considered.

\begin{enumerate}[start=3]
	\item Is it necessary to audit the system's provenance?
	\item Is it necessary to prevent malicious data deletion?
\end{enumerate}

If auditability and data replication are critical, blockchain technology should be considered. This is due to the fact that meaningful shared governance \emph{and} operation requires miners to audit the operations of others and to be able to recover data that a malicious miner might try to delete.

Even though Blockchain technology does not solve all the problems that its proponents claim it does, we believe that it is a meaningful technology that will continue to be used in industry and is deserving of further research and experimentation.

