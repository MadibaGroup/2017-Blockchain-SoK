% !TEX root = ../main.tex

\section{Conclusion}

Many proponents of Blockchain technology believe that it has the capability to massively disrupt how society operates, or at least to rapidly overtake legacy solutions in many significant industries. This belief is hyperbole (i.e., hype) as though Blockchain technology has many valid uses, it has not, nor is it likely to achieve this Utopian vision. This ideology and hype cause problems: for example, frequent emotionally-charged schisms within Blockchain advocate and developer communities---especially those affiliated with Bitcoin. This turmoil prevents level-headed scientific discourse and wastes developer resources. It can also tangibly affect the stability of a Blockchain systems by causing a fork, in which two independent chains emerge to used and maintained by different groups, further dividing resources.

With that said, Blockchain technology's disruptive power has certainly been demonstrated in the financial sector, so it clearly has promise. Several factors have made this sector an attractive target for disruption, perhaps none more so than the opportunity for massive profit. This motive has had benefits for Blockchain technology, especially in accelerating the pace of technological development. However, it has also created perverse incentives to reinforce hype and ideology. Hype can attract investors and inflate valuations, and dogmatic ideology is a proven marketing and recruitment strategy for financial scammers. These problems inhibit the advancement of Blockchain technology.


