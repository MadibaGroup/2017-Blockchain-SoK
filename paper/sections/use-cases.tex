% !TEX root = ../main.tex

\section{Use Cases}
\label{sec:use-cases}

Industry and government can apply blockchain in a number of uses cases which 
require shared governance, verifiable state, and/or resilience to data loss. 

%Within the literature we analyzed, there was significant discussion about potential applications for Blockchain technology.
%We describe these use cases here.
%In many cases, an argument could be made that these are not independent use cases. Some use cases are subsumed by others, or two use cases might have a common abstraction that make them effectively the same.
%As authors, we attempt to categorize in a way that is useful to the reader trying to locate information about a particular use case. Before discussing the use cases themselves, we first discuss the general parameters of when a Blockchain is useful. 

\subsection{Financial Use Cases}

\paragraph{Electronic currencies and payments}
% Jeremy: previously: cryptocurrencies and electronic payments

% Jeremy: add citations
% low-latency: ? HoneyBadger? Collective signalling?
% scalable: Omniledger, scaling position paper, Bitcoin-NG
% partially offline: payment channels, lightning
% confidential: CT bulletproofs
% anonymous: zerocoin, zerocash, etc

It is well-known that blockchain technology can be used to build cryptocurrencies; Bitcoin is a working example of this. Blockchain technology enables electronic transactions that are resilient even when large amounts of money are at stake. Bitcoin has notable drawbacks that include low scalability, high energy consumption, and merely moderate privacy protections. However a consortium payment system on a permissioned blockchain can address the first two key challenges. 

%There is ongoing research that demonstrates how Blockchain technology can be used to create payment systems that are low-latency and scalable~\cite{}, partially offline~\cite{}, confidential~\cite{} and/or anonymous~\cite{}.
%There is ongoing research that demonstrates how Blockchain technology can be used to create payment systems that are low-latency and scalable, partially offline, confidential, and/or anonymous (see Appendix~\ref{sec:academic}).

\paragraph{Asset trading}
% Jeremy: previously Token tracking and trading
% Jeremy: changed to clarify differences with next one

Financial markets allow the exchange of assets. They tend to involve intermediaries like exchanges, brokers and dealers, depositories and custodians, and clearing and settlement entities. Blockchain-based assets---which are either intrinsically valuable or are a claim on an off-chain asset (material or digital)---can be transacted directly between participants, governed by smart contracts that can provide custodianship, and require less financial market infrastructure. Two key challenges are (1) stapling for tokens that represent something off-chain (\ie equity in a firm or a debt instrument), and (2) government oversight and regulatory compliance.
% Jeremy: add LVTS

\paragraph{Markets and auctions}

A central component of asset trading is the market itself---the coordination point for buyers and sellers to find each other, exchange assets, and provide price information to observers.
Auctions are a common mechanism for setting a fair price; this includes double-sided auctions like the order books in common use by financial exchanges. 
%Decentralized markets and auction mechanisms can run on blockchain technology.
The key challenge for a decentralized market is that transactions are broadcast to the consensus protocol and thus non-confidential, hindering privacy and enabling front-running.
%An additional challenge is the potential for front-running by participants in the Blockchain network (in particular, miners) who learn of transactions before they are finalized. 

\paragraph{Insurance and futures}

Transactions can be arranged that are contingent on future times or events. Examples include a purchase of assets at a future time for a locked-in price, an insurance payout for a fire, or action on a loan default. The key challenges are (1) determining trustworthy oracles to report relevant off-chain events like fires, exchange rates, \etc (or limit the contract to on-blockchain events) and (2) choosing between a design that locks up so much collateral it can settle all possible eventualities, or a leaner design where the counterparty promises to fulfill its obligations but there is the \textit{counterparty risk} it will not.

%The primary challenge of transactions of this the risk that the counterparty will not fulfill their future obligations. While Blockchain technologies can reduce some types of trust, it cannot easily solve counterparty risk. 
%It can offer transparency which can be used to build reputation and contracts can be designed to hold digital currency or assets as collateral and disperse them if Blockchain-based conditions are met.
%A second challenge for any of these use cases is reporting real world events (a fire, a change in the price of a stock, or mortgage default) to the Blockchain in a trustworthy fashion.
%However, this is not an issue for events that are Blockchain-based to begin with.
%This demonstrates a key point: deploying several complimentary use cases on the same Blockchain enables complex interactions. 


\paragraph{Penalties, remedies, and sanctions}

%In common parlance, code running on a Blockchain is called a smart contract.\footnote{In particular, the success of Ethereum contributed to this, although that project now prefers the term `decentalized app' or dapp for short.}

Legal contracts anticipate potential future breaches and specify a set of penalties or remedies. With blockchain technology, remedies for likely outcomes could be programmed (these could still be later overturned through traditional litigation). Similar to insurance and futures, oracles and counter-party risk are key challenges.

%If the contract is legally well-formed, with identified counterparties in a clear jurisdiction, the remedies can be thought of as a set of reasonable default actions that can avoid, but do not preclude, expensive litigation. 

% =====

\subsection{Data Storage and Sharing Use Cases}

\paragraph{Asset tracking} 
% Jeremy: Merger of separate sections ``asset tracking'' and ``supply chain management''

Blockchain technology can be used to track material assets that are globally distributed, valuable, and whose provenance is of interest. This includes standalone items like artwork and diamonds, certified goods like food and luxury items, dispersed items like fleets of vehicles, and packages being shipped over long distances, which will change hands many times in the process. 
It also includes the individual components of complex assembled devices, where the parts originate from different firms. 
For heavily regulated industries, like airlines, and for military/intelligence applications, it is important to establish the source of each part that has been used, as well as a maintenance history (\ie its provenance).
%While assets are already tracked in digital databases, there is no common database shared by each participant in the supply chain.  
%
%A Blockchain sidesteps the political problem of who should host such a shared database when the candidates are competing firms and government agencies from different jurisdictions. 
Blockchain technology provides a common environment where no single firm has the elevated power and control of running the database that tracks this information. Key challenges are the reliable stapling of data, confidentiality, and on-boarding all the necessary firms onto the same blockchain.  
%The main integrity challenge is the stapling issue: specifying how material assets are assigned a tracking token on the Blockchain in a trustworthy manner. 
%A second challenge is the lack of confidentiality Blockchain technologies offers by default when the data is proprietary and tied to profitable business practices.
%Finally, a third challenge is getting agreement on the technology to be used (this is being explored through business consortiums.\footnote{\eg Blockchain in Transport Alliance}).

\paragraph{Identity and key management}

Identities, along with cryptographic attestations about properties for those identities (\eg over 18 years of age, has a driver's license, owns a specific cryptographic key) can be maintained on a blockchain. This is a special case of asset tracking (above), where the asset is a person and the key challenges are the same.

\paragraph{Tamper-resistant record storage}

The append-only ledger of a blockchain can be used to store documents, including the history of changes to these documents. This use case is best suited for records that are highly valuable (such as certificates, government licenses), have a small data size, and are publicly available (as they will be replicated by all miners). If large and/or confidential documents need to be stored, then a blockchain might store secure pointers (\ie binding/hiding commitments) for the documents, while the documents themselves are stored in a different system.

% ===

\subsection{Other use cases}

\paragraph{Voting}
Electronic voting is a challenging problem that is often asserted to benefit from blockchain technology's properties.
Shared governance could be used to ensure that multiple parties (the government, non-governmental organizations, international watchdogs) can all work together to ensure that an election is legitimate.
Audibility is important in providing evidence to the electorate that the election was fair.
Finally, the resilience of blockchain technology is important in preventing cyberattacks against the voting system.
However, voting on a blockchain has many challenges to solve: blockchains offer no inherent support for secret ballots, electronic votes can be changed by the device from which they are submitted (undetectably if a secret ballot is achieved), cryptographic keys could be sold to vote buyers, and key recovery mechanisms would need to be established for lost keys. 

\paragraph{Gambling and Games}
Gambling is very popular on Bitcoin and Ethereum already. Players can audit the contract code to ensure that execution is fair, and the contract can use cryptocurrency to handle the finances (including holding the money in escrow to prevent losing parties from aborting before paying). This use case is best suited to gambling games that do not require randomness, private state, or knowledge of off-blockchain events.

%For the set of residual games,\footnote{Currently, the most active Ethereum game is called Fomo3D: users pay to reset a 30 second countdown timer and if it ever reaches zero, the last user to pay wins all the money collected.} Blockchain is an ideal platform. For the other types of games, new layers of technology would have be added on a Blockchain. Data feeds (called oracles) of either randomness or real world event outcomes requires additional trust and introduces finality risk, while confidential user state require additional cryptography.  

%\paragraph{IoT and smart property}
%% Jeremy: To do
%IoT devices occasionally have the need to collectively make decisions.
%In these cases, Blockchain technology can provide a technological platform for making these collective decisions in an auditable fashion.
%This auditability is especially important as IoT devices are notoriously untrustworthy due to insufficient security, and the ability to later audit and analyze their actions is invaluable.
%Additionally, the replication inherent to Blockchain technology means that even if a subset of the IoT devices is lost (e.g., destroyed sensors in a storm), it can still be possible to record the entire provenance of all devices.

% Jeremy: Removing below for now. While I don't mind including bad use cases (see voting), this one is not well defined and is vague

%\paragraph{Interoperation}
%Blockchain technology can facilitate the interoperation of multiples systems.
%It can provide a place for multiple systems to write their data and make requests of other systems.
%Using smart contracts it can convert those requests to the appropriate form to allow them to be parsed by other systems.
%If a central database server is not suitable for the application, this use case may be valid, but in most cases this can more easily be accomplished with a standard program and a database.
%%This is a poor use case because it does not rely on shared governance, consensus, or provenance.

