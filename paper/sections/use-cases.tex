% !TEX root = ../main.tex

\section{Blockchain Technology's Use Cases}
\label{sec:use-cases}

%\textbf{TODO: Insert references, especially for each class}
%\textbf{TODO: discuss how we count applications.  I've called several things to be special cases of others--ok?}

%List application areas
%	Why are they good
%	Interesting research questions

%The enthusiasm that many start-ups have in leveraging Blockchain technology is justified, as the range of use cases is broad.  Good use cases leverage sets of capabilities that Blockchain technology can provide, using technological properties in combinations that are unique or nearly unique to Blockchain technology.  For these companies, decentralized governance is an important element of the blockchain, because it diffuses trust amongst those who maintain the blockchain.  This increases resilience by reducing the ability of a single bad actor from modifying or damaging the blockchain, and enables recovery when one bad actor starts to attempt these actions.  A second valuable property comes from the ledger, which records initial state and transitions, thereby providing a history of the provenance of the recorded data.  Many commercial companies understand this and consider these properties when selecting the blockchain.  

%However, we find that it is just as important to consider the challenges and limitations of using the blockchain, because these limit the applications built on top of it in ways that are important to industry.  Some of these limit the ability of a business to meet regulations and to respond to legal challenges, and others limit the growth of the company or the ability of the company to handle large-scale problems.  One misconception we see is the idea that pseudonymity provides anonymity or privacy.  It alone does not; however, it is possible to leverage additional cryptography to achieve this feature.  We find that since only a portion of applications rely on this, it is best to treat anonymity as a separate, add-on property.

%Finally, we find that the hype around the blockchain has led to some companies pursuing it, even if other, more appropriate capabilities and data structures are available.  One example flows from capabilities provided by the ledger.  Applications that benefit from maintaining a history of transactions are a good fit for the blockchain, as the current state can be reconstituted by traversing the append-only transaction ledger.  If however, the application only needs to know the current state without the history of prior states, then more efficient distributed data stores exist and should probably be used.

Based on the results of our analysis, we have identified sixteen classes of applications for Blockchain technology.
These use cases are summarized in Table~\ref{tab:usecase}.

\begin{table}
\renewcommand{\arraystretch}{1.35}
\setlength\tabcolsep{.25em}
\centering 

\caption{Blockchain Technology Use Cases and Associated Capabilities and Challenges.\label{tab:usecase}}

\begin{tabular}{ll | *{8}{c} | *{5}{c} |}
	
	\headrow{} & \headrow{} \headline &
	
	\headrow{Shared gov. and op.} &
	\headrow{Resilience} &
	\headrow{Provenance} &
	\headrow{Auditability} &
	\headrow{Access control} &
	\headrow{Pseudonymity} &
	\headrow{Smart contracts} &
	\headrow{Data discoverability}
	\headline &
	
	\headrow{Scalability} &
	\headrow{Off-chain} &
	\headrow{Correctness} &
	\headrow{Key management} &
	\headrow{Privacy}
	\headline \\ \hline
	
	& \textit{Use Case} & 
	\multicolumn{8}{c|}{\textit{Capabilities}}&    
	\multicolumn{5}{c|}{\textit{Challenges}} \\ \hline
	
	&Asset Management
	&\none	&\none	&\none	&\none	&\none	&\none	&\none	&\none
	&\none	&\none	&\none	&\none	&\none	\\
	
	&\hspace{1em} Supply chain management
	&\none	&\none	&\none	&\none	&\none	&\none	&\none	&\none
	&\none	&\none	&\none	&\none	&\none	\\
	
	&\hspace{1em} Asset tracking
	&\none	&\none	&\none	&\none	&\none	&\none	&\none	&\none
	&\none	&\none	&\none	&\none	&\none	\\
	
	&\hspace{1em} Payments
	&\none	&\none	&\none	&\none	&\none	&\none	&\none	&\none
	&\none	&\none	&\none	&\none	&\none	\\
	
	&\hspace{1em} Transaction processing
	&\none	&\none	&\none	&\none	&\none	&\none	&\none	&\none
	&\none	&\none	&\none	&\none	&\none	\\
	
	\hline
	
	&Insurance
	&\none	&\none	&\none	&\none	&\none	&\none	&\none	&\none
	&\none	&\none	&\none	&\none	&\none	\\
	
	&IoT / smart property
	&\none	&\none	&\none	&\none	&\none	&\none	&\none	&\none
	&\none	&\none	&\none	&\none	&\none	\\
	
	\hline
	
	&Regulation / sanctions
	&\none	&\none	&\none	&\none	&\none	&\none	&\none	&\none
	&\none	&\none	&\none	&\none	&\none	\\
	
	&Record storage
	&\none	&\none	&\none	&\none	&\none	&\none	&\none	&\none
	&\none	&\none	&\none	&\none	&\none	\\
	
	\hline
	
	&Key management
	&\none	&\none	&\none	&\none	&\none	&\none	&\none	&\none
	&\none	&\none	&\none	&\none	&\none	\\
	
	&Identity management
	&\none	&\none	&\none	&\none	&\none	&\none	&\none	&\none
	&\none	&\none	&\none	&\none	&\none	\\
	
	\hline
	
	&Fine-grained access control
	&\none	&\none	&\none	&\none	&\none	&\none	&\none	&\none
	&\none	&\none	&\none	&\none	&\none	\\
	
	&Data sharing
	&\none	&\none	&\none	&\none	&\none	&\none	&\none	&\none
	&\none	&\none	&\none	&\none	&\none	\\

	&System interoperability
	&\none	&\none	&\none	&\none	&\none	&\none	&\none	&\none
	&\none	&\none	&\none	&\none	&\none	\\
	
	\hline
	
	&Voting
	&\none	&\none	&\none	&\none	&\none	&\none	&\none	&\none
	&\none	&\none	&\none	&\none	&\none	\\
	
	&Auction / bidding
	&\none	&\none	&\none	&\none	&\none	&\none	&\none	&\none
	&\none	&\none	&\none	&\none	&\none	\\
	
	&Gambling
	&\none	&\none	&\none	&\none	&\none	&\none	&\none	&\none
	&\none	&\none	&\none	&\none	&\none	\\

	&Forecasting
	&\none	&\none	&\none	&\none	&\none	&\none	&\none	&\none
	&\none	&\none	&\none	&\none	&\none	\\

	\hline

	&Decentralized timestamping
	&\none	&\none	&\none	&\none	&\none	&\none	&\none	&\none
	&\none	&\none	&\none	&\none	&\none	\\
	
	\hline

\end{tabular}
\end{table}


% === 

\subsection{Ideal Use Cases}
In this subsection, we describe the use cases that could benefit the most from the application of Blockchain technology.

\subsubsection{Payments}
It is well-known that Blockchain technology can be used to build cryptocurrencies, thus enabling electronic payments.
Bitcoin is a working example of this, however Bitcoin has notable drawbacks that include limited throughput and privacy.
There is ongoing research that demonstrates how Blockchain technology can be used to create payment systems that are low-latency, scalable, partially offline, confidential and/or anonymous.

\subsubsection{Transaction Processing} % Jeremy: Should rename to tokens or something like that
Real world financial markets provide the ability for the exchange of valuable assets and tend to involve intermediaries like exchanges, brokers or dealers, clearing and settlement entities. 
Blockchain-based assets---which either hold intrinsic value on it's own blockchain, or represent a claim on an external material or digital asset---can be transacted directly between participants or governed by smart contracts.
Similar to payments, such transactions are resilient even when large values are at stake.
Shared governance enables direct trading between users and requires less financial market infrastructure.  

\subsubsection{Auctions/Markets}
A Blockchain-based market can provide a coordination point for buyers and sellers to find each other, exchange assets, and provide price information to observers. 
Auctions are a common mechanism for setting a fair price. 
Decentralized markets and auction mechanisms must address the issue of non-confidential transactions and the resulting potential for front-running by participants in the blockchain network (in particular, miners).

\subsubsection{Record Storage} % Jeremy: Data Provenance
A Blockchain system's append-only ledger can be used to store documents, including the history of changes to these documents.
This use case is best suited for records that are highly valuable (such as certificates, government licenses), that have a small data size (storage is limited and expensive in most blockchain systems), and are publicly available (by default, a blockchain does not offer confidentiality).
A common mechanism to expand blockchain storage to large and/or confidential data is to only store succinct binding/hiding commitments to the data on the blockchain, and store the raw data itself in a less integral environment.
For open Blockchains, the identity of the entity certifying the data is not provided for by default on a Blockchain.

% === 

\subsection{Suitable Use Cases}
In this subsection, we describe the use cases that could benefit from Blockchain technology if there is a need for shared governance, provenance, and/or high-level reliability. Each of these use-cases include at least one major unresolved issue.

\subsubsection{Asset Tracking} 
Blockchain technology can be used for tracking material assets that are globally distributed, valuable, and whose provenance is of interest, when resilience and audibility are important. This includes long-term valuable items like artwork and diamonds; certified goods like food and luxury items; dispersed items like fleets of vehicles; and short-term items like package being shipped over long distances, which will change hands many times in the process. While assets are already tracked through digital databases, a blockchain solves the problem of who should run the database for industries with multiple participants. It provides a common environment where no single firm has the elevated power and control of running a widely-used database. The main challenge is specifying how material assets are assigned a tracking token on the blockchain in a trustworthy manner. A second challenge is the level of transparency a blockchain would bring to proprietary, profitable business practices. 

% Analyses of interest include identifying the last known locations and durations that the asset spends in specific places or transiting between places.

\subsubsection{Supply Chain Management} % Jeremy: needs better name
Many complex devices built today require parts that are built by multiple companies, assembled into a useful whole. For heavily regulated industries, like airlines, or for military/intelligence applications, it is important to establish the source of each part that has been used, as well as a maintenance history. The benefit of a blockchain, like asset tracking, is mainly a political one (not a technical one): it enables a common environment while side-stepping the question of which entity maintains the environment. It also shares the same binding challenge.

%To support this, asset tracking needs to be augmented with a mechanism enabling multiple items to be combined into a single new item.   Here information can distributed: multiple suppliers could be creating the same parts, and multiple assembly plants can remove those parts from inventory and produce the assembled products.  An important element of the profitability of a company comes from careful management of this supply chain and the inventory that is developed by the companies.  Thus analysis needs to support the inventory turnover metric which is used to determine the quality of supply chain management.

\subsubsection{Identity Management}
Identities along with cryptographic attestations about properties for those identities (e.g., over 18 years of age, has a driver's license) can be written to the Blockchain.
These identities and attestations can then be used by other systems to support their access control policies.
Importantly, this identity information comes with full provenance. This could be useful in determine suspicious activity (e.g., having an age that is not increasing linearly).
This could also be a quicker and more performant way of establishing identity than the current certificate authority system.

\subsubsection{Gambling}
By examining the most active Bitcoin scripts and Ethereum decentralized applications, gambling is popular. Players can audit the game to ensure that execution is fair, and the system can operate its own cryptocurrency to handle the finances (including holding the money in escrow to prevent losing parties from aborting before paying). This use case is best suited to gambling games that do not require randomness, private state, or knowledge of off-blockchain events. For the subset of residual games,\footnote{The most active Ethereum game is called Fomo3D: users pay to reset a 30 second countdown timer and if it ever reaches zero, the last user to pay wins all the money collected.} blockchain is an ideal platform. For the other types of games, new layers of technology would have be added on a Blockchain. Data-feeds (called oracles) of either randomness or real world event outcomes requires additional trust and introduces finality risk, while confidential user state require additional cryptography.  

% Jeremy: Watermark for me when I pick this up again. 

\subsubsection{IoT/Smart Properties}
IoT devices occasionally have the need to collectively make decisions.
In these cases, Blockchain technology can provide a technological platform for making these collective decisions in an auditable fashion.
This auditability is especially important as IoT devices are notoriously untrustworthy due to insufficient security, and the ability to later audit and analyze their actions is invaluable.
Additionally, the replication inherent to Blockchain technology means that even if a subset of the IoT devices is lost (e.g., destroyed sensors in a storm), it can still be possible to record the entire provenance of all devices.

\subsubsection{Insurance}
Insurance policies can be recorded on the Blockchain.
As off-chain oracles report on accidents and other insured events, the system can automatically begin the claim process and even pay out the insurance money upon completion of the claim.
The auditability provides assurances to customers.
Additionally, having insurers collectively govern and operate this system could ensure that claims are properly settled when they involve multiple insurers.

\subsubsection{Regulation/Sanctions}
Much like the insurance use cases, regulations can be written to a Blockchain and off-chain oracles can report on real-world violations of those regulations.
After a violation is reported, a smart contract can automatically apply the penalty or sanction for breaking the regulation.
As the system is auditable, penalized parties have a full history of what led to their punishment, providing means for them to appeal if there was a mistake.

\subsubsection{Fine-grained Access Control}
In practice, access control systems are often centralized. They are sometimes nominally distributed -- for example, requiring several written signatures on a form before a system administrator will grant a new permission -- but even these systems are essentially centralized as a single administrator can unilaterally grant nearly any privilege in the system.
Blockchain technology's shared governance could allow for true fine-grained and decentralized access control.
When permissions are needed, those responsible (and not an administrator) can directly approve the permissions and smart contracts can ensure that the permissions is granted if and only if all necessary approvals are given.
Additionally, the provenance information associated with the access control can be used to rollback changes made by users if those users are later shown to have been compromised.

\subsubsection{Data Sharing}
One key challenge when sharing data between multiple organizations is that no organization wants to give up ownership and control of their own data.
Blockchain technology's shared governance model allows all parties to share data while still retaining control of their own data.
Importantly, it is not the data that is stored on the Blockchain, but rather the access control rules and access attempts.
The data itself continues to be stored by the organization that owns/controls it.

%=== 
\subsection{Unsuitable Use Cases}
In this subsection, we describe uses cases that could use Blockchain technology, but most likely should not.

\subsubsection{Voting}
Electronic voting is a challenging problem which might benefit from several of Blockchain technology's properties.
Shared governance could be used to ensure that multiple parties (the government, non-governmental organizations, international watchdogs) can all work together to ensure that an election is legitimate.
Audibility is important in providing evidence to the electorate that the election was fair.
Finally, the resilience of Blockchain technology is important in prevent cyber-attacks against the voting system.

\subsubsection{Interoperation}
Blockchain technology can facilitate the interoperation of multiples systems.
It can provide a place for multiple systems to write their data and make requests of other systems.
Using smart contracts it can convert those requests to the appropriate form to allow them to be parsed by other systems.\
This is a poor use case because it does not rely on shared governance, consensus, or provenance.
This use case can more easily be accomplished with a standard program and a database.



\subsubsection{Timestamping}
Blockchain technology could be used to timestamp documents. Still, timestamping generally does not rely on any of Blockchain technologies key properties, and so Blockchain technology is likely overkill for this application.