% !TEX root = ../main.tex

\section{Blockchain Technology's Use Cases}
\label{sec:use-cases}

Within the literature we analyzed, there was significant discussion about potential applications for Blockchain technology.
We describe these use cases here.
In many cases, an argument could be made that these are not independent use cases. Some use cases are subsumed by others, or two use cases might have a common abstraction that make them effectively the same.
As authors, we attempt to categorize in a way that is useful to the reader trying to locate information about a particular use case. Before discussing the use cases themselves, we first discuss the general parameters of when a Blockchain is useful. 

\subsection{Relationship to other distributed systems}

Blockchain technology provides a unique set of capabilities that might be better suited for a system design than competing database technologies or distributed systems.
However it does come with a relatively high overhead: replicating all past and present data and operations on the data at every node of the network in an auditable ledger.
Based on our results and experience we recommend the use of the following questions to determine if Blockchain technology would be a good fit for a specific project (see Section~\ref{sec:sharedgov} for definitions).\footnote{In Appendix~\ref{sec:distributed-comparison} we discuss alternatives to Blockchain technology.}

\begin{enumerate}
	\item Does the system require shared governance?
	\item Does the system require shared operation?
\end{enumerate}

If the answer to both questions is no, then Blockchain's consensus protocol is likely unnecessary overhead. If the answer to both questions is yes, then Blockchain technology is likely a good fit. This is due to the fact that meaningful shared governance \emph{and} operation requires miners to audit the operations of others and to be able to recover data that a malicious miner might try to delete (questions 3 and 4 below, respectively). If only shared governance or shared operation is needed, then the following two questions can be used to determine if the auditable ledger and replication, respectively, justifying the use of Blockchain technology if both are needed:

\begin{enumerate}[start=3]
	\item Is it necessary to audit the system's provenance?
	\item Is it necessary to prevent malicious data deletion?
\end{enumerate}



% =====

\subsection{Financial Use Cases}

\paragraph{Electronic currencies and payments}
% Jeremy: previously: cryptocurrencies and electronic payments

% Jeremy: add citations
% low-latency: ? HoneyBadger? Collective signalling?
% scalable: Omniledger, scaling position paper, Bitcoin-NG
% partially offline: payment channels, lightning
% confidential: CT bulletproofs
% anonymous: zerocoin, zerocash, etc

It is well-known that Blockchain technology can be used to build cryptocurrencies; Bitcoin is a working example of this.
Blockchain technology enable electronic transactions that are resilient even when large values are at stake.
Bitcoin has notable drawbacks that include scalability, performance and privacy.
%There is ongoing research that demonstrates how Blockchain technology can be used to create payment systems that are low-latency and scalable~\cite{}, partially offline~\cite{}, confidential~\cite{} and/or anonymous~\cite{}.
There is ongoing research that demonstrates how Blockchain technology can be used to create payment systems that are low-latency and scalable, partially offline, confidential, and/or anonymous (see Appendix~\ref{sec:academic}).

\paragraph{Asset trading}
% Jeremy: previously Token tracking and trading
% Jeremy: changed to clarify differences with next one

Real world financial markets provide the ability for the exchange of valuable assets. 
They tend to involve intermediaries like exchanges, brokers and dealers, depositories and custodians, and clearing and settlement entities. 
Blockchain-based assets---which are either intrinsically valuable or are a claim on an off-chain asset (material or digital)---can be transacted directly between participants, governed by smart contracts that can provide custodianship, and require less financial market infrastructure.
For tokens (i.e., digital identifiers) that represent something off-chain (\ie equity in a firm or a debt instrument), the issue of stapling must be addressed.
In most jurisdictions, financial markets are subject to government oversight making this area particularly encumbered by regulation.
% Jeremy: add LVTS

\paragraph{Markets and auctions}

A central component of asset trading is the market itself---the coordination point for buyers and sellers to find each other, exchange assets, and provide price information to observers.
Auctions are a common mechanism for setting a fair price; this includes double-sided auctions like the order books in common use by financial exchanges. 
Decentralized markets and auction mechanisms can run on Blockchain technology.
The main challenge is the issuance of non-confidential transactions. 
An additional challenge is the potential for front-running by participants in the Blockchain network (in particular, miners) who learn of transactions before they are finalized. 

\paragraph{Insurance and futures}

A transaction can be thought of as a swap of one valuable asset for another. Swaps can be arranged for a future time.
Examples include agreeing to a price for a future purchase of oil, turning a variable interest rate into a fixed rate, and providing cash for collateral with the promise that the cash will be returned.
Further, this future swap might be contingent on an event happening. 
Examples include a derivative that is valuable if a stock price decreases, an insurance payout for a fire, or a payment that covers a loan default.
The primary challenge of transactions of this the risk that the counterparty will not fulfill their future obligations. While Blockchain technologies can reduce some types of trust, it cannot easily solve counterparty risk. 
It can offer transparency which can be used to build reputation and contracts can be designed to hold digital currency or assets as collateral and disperse them if Blockchain-based conditions are met.
A second challenge for many of these use cases is reporting real world events (a fire, a change in the price of a stock, or mortgage default) to the Blockchain in a trustworthy fashion.
However, this is not an issue for events that are Blockchain-based to begin with.
This demonstrates a key point: deploying several complimentary use cases on the same Blockchain enables complex interactions. 


\paragraph{Penalties, remedies, and sanctions}

In common parlance, code running on a Blockchain is called a smart contract.\footnote{In particular, the success of Ethereum contributed to this, although that project now prefers the term `decentalized app' or dapp for short.} Legal contracts will often anticipate potential future breaches and offer a set of penalties or remedies.
With Blockchain technology, a set of remedies could be programmed into the contract assuming both the triggering event and the resulting action are Blockchain-based (or the real-world/Blockchain gap is bridged by a trustworthy entity).
If the contract is legally well-formed, with identified counterparties in a clear jurisdiction, the remedies can be thought of as a set of reasonable default actions that can avoid, but do not preclude, expensive litigation. 

% =====

\subsection{Data Storage and Sharing Use Cases}

\paragraph{Asset tracking} 
% Jeremy: Merger of separate sections ``asset tracking'' and ``supply chain management''

Blockchain technology can be used for tracking material assets that are globally distributed, valuable, and whose provenance is of interest.
This includes standalone items like artwork and diamonds, certified goods like food and luxury items, dispersed items like fleets of vehicles, and packages being shipped over long distances, which will change hands many times in the process. 
It also includes the individual components of complex assembled devices, where the parts originate from different firms. 
For heavily regulated industries, like airlines, and for military/intelligence applications, it is important to establish the source of each part that has been used, as well as a maintenance history.
While assets are already tracked in digital databases, there is no common database shared by each participant in the supply chain.  

A Blockchain sidesteps the political problem of who should host such a shared database when the candidates are competing firms and government agencies from different jurisdictions. 
Blockchain technology provides a common environment where no single firm has the elevated power and control of running a widely-used database. 
The main integrity challenge is the stapling issue: specifying how material assets are assigned a tracking token on the Blockchain in a trustworthy manner. 
A second challenge is the lack of confidentiality Blockchain technologies offers by default when the data is proprietary and tied to profitable business practices.
Finally, a third challenge is getting agreement on the technology to be used (this is being explored through business consortiums.\footnote{\eg Blockchain in Transport Alliance}).

\paragraph{Identity and key management}
% Jeremy: To do still

Identities, along with cryptographic attestations about properties for those identities (e.g., over 18 years of age, has a driver's license), can be written to the Blockchain.
These identities and attestations can then be used by other systems to support their access control policies.
Importantly, this identity information comes with full provenance. This could be useful in determining suspicious activity (e.g., having an age that is not increasing linearly).
This could also be a quicker and more performant way of establishing identity than the current certificate authority system.

\paragraph{Multi-organization data sharing}

Asset tracking and identity tracking are both examples of sharing data across organizations, and Blockchain technology contributes a common environment.
The use cases in this category share challenges: Blockchain technology can specify write access policies to data stored on the Blockchain, but it will not provide any default support for restricting read access; confidentiality has to be an additional layer.
Further, the integrity of confidential data cannot be validated by nodes in the network without some minimal disclosure of what the data is or that it satisfies the relevant restrictions.
Blockchain can also serve as a secondary component in these systems, where capabilities are issued and transferred as if they were financial assets on a Blockchain. Proving ownership over a capability is done off-chain to a traditional enforcement server, which enables the correct permissions.


\paragraph{Tamper-resistant record storage}
% Jeremy: To do
% Jeremy: Merge of permanent record storage and timestamping
% Jeremy: Haber Stornetta
The append-only ledger is used to store documents, including the history of changes to these documents.
This use case is best suited for records that are highly valuable (such as certificates, government licenses), have a small data size, and are publicly available (as they will be replicated by all miners).
If large and/or confidential documents need to be stored, then the Blockchain can store binding/hiding commitments for the documents, while the documents themselves are stored in another system with lower overhead.
%For open Blockchains, the identity of the entity certifying the data is not provided for by default on a Blockchain.
% Jeremy: Querying -> give me everything
Blockchain technology could be used to timestamp documents. Still, timestamping generally does not rely on any of Blockchain technology's key properties, and so Blockchain technology is likely overkill for this application.


% Jeremy: Why is this removed? I don't disagree with its removal but want to confirm the motivation

%\subsubsection{Fine-grained Access Control}
%In practice, access control systems are often centralized. They are sometimes nominally distributed -- for example, requiring several written signatures on a form before a system administrator will grant a new permission -- but even these systems are essentially centralized as a single administrator can unilaterally grant nearly any privilege in the system.
%Blockchain technology's shared governance could allow for true fine-grained and decentralized access control.
%When permissions are needed, those responsible (and not an administrator) can directly approve the permissions and smart contracts can ensure that the permissions is granted if and only if all necessary approvals are given.
%Additionally, the provenance information associated with the access control can be used to rollback changes made by users if those users are later shown to have been compromised.

% ===

\subsection{Other use cases}

\paragraph{Voting}
% Jeremy: To do
Electronic voting is a challenging problem that is often asserted to benefit from Blockchain technology's properties.
Shared governance could be used to ensure that multiple parties (the government, non-governmental organizations, international watchdogs) can all work together to ensure that an election is legitimate.
Audibility is important in providing evidence to the electorate that the election was fair.
Finally, the resilience of Blockchain technology is important in preventing cyberattacks against the voting system.
However, voting on a Blockchain has many challenges to solve: Blockchains offer no inherent support for secret ballots, electronic votes can be changed by the device from which they are submitted (undetectably if a secret ballot is achieved), cryptographic keys could be sold to vote buyers, and key recovery mechanisms would need to be established. 

\paragraph{Gambling and Games}
% Jeremy: To improve

Examining the most active Bitcoin scripts and Ethereum decentralized applications shows that gambling is quite popular. Players can audit the game to ensure that execution is fair, and the system can operate its own cryptocurrency to handle the finances (including holding the money in escrow to prevent losing parties from aborting before paying). This use case is best suited to gambling games that do not require randomness, private state, or knowledge of off-blockchain events. For the set of residual games,\footnote{Currently, the most active Ethereum game is called Fomo3D: users pay to reset a 30 second countdown timer and if it ever reaches zero, the last user to pay wins all the money collected.} Blockchain is an ideal platform. For the other types of games, new layers of technology would have be added on a Blockchain. Data feeds (called oracles) of either randomness or real world event outcomes requires additional trust and introduces finality risk, while confidential user state require additional cryptography.  

\paragraph{IoT and smart property}
% Jeremy: To do
IoT devices occasionally have the need to collectively make decisions.
In these cases, Blockchain technology can provide a technological platform for making these collective decisions in an auditable fashion.
This auditability is especially important as IoT devices are notoriously untrustworthy due to insufficient security, and the ability to later audit and analyze their actions is invaluable.
Additionally, the replication inherent to Blockchain technology means that even if a subset of the IoT devices is lost (e.g., destroyed sensors in a storm), it can still be possible to record the entire provenance of all devices.

% Jeremy: Removing below for now. While I don't mind including bad use cases (see voting), this one is not well defined and is vague

%\paragraph{Interoperation}
%Blockchain technology can facilitate the interoperation of multiples systems.
%It can provide a place for multiple systems to write their data and make requests of other systems.
%Using smart contracts it can convert those requests to the appropriate form to allow them to be parsed by other systems.
%If a central database server is not suitable for the application, this use case may be valid, but in most cases this can more easily be accomplished with a standard program and a database.
%%This is a poor use case because it does not rely on shared governance, consensus, or provenance.

