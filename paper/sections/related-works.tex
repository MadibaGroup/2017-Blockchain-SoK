% !TEX root = ../main.tex

\section{Related Works (Ben))}
Several prior authors have surveyed and systematized different aspects of the Blockchain landscape.
One of the most comprehensive works is from Bonneau et al., focusing on Bitcoin and other cryptocurrencies~\cite{BMC+15}. The authors seek to understand how Bitcoin has worked successfully and securely in practice for several years despite the lack of a rigorous model for its correctness and security. From a host of disparate sources, they distill a clear technical overview of how Bitcoin works. They then systematize the properties sufficient for system stability and analyze how they hold up under different assumptions about the incentives for miners. Finally, they evaluate and compare proposed changes to Bitcoin including tweaks to system parameters, alternative consensus protocols, and improvements to privacy and anonymity and conclude with a discussion of additional functionalities that Bitcoin could be extended to provide.  

Their approach differs from ours in several key ways, reflecting that our analyses have fundamentally different goals. They focus on a single application of Blockchain technology -- cryptocurrencies -- and survey technical sources in order to analyze how systems work and how design changes might affect functionality. By instead surveying sources that analyze Blockchain, we systematize the collective knowledge of the community to understand how Blockchain is being thought about and used by practitioners. This approach allows us to reason about how Blockchain may be useful in novel scenarios and to separate out normative beliefs from technical ones. 

Bano et al. focus on a particular technical component of Blockchain: the consensus process \cite{Bano17}. They systematize different protocols for consensus into three categories, identify shared characteristics and high-level design themes, and provide an evaluation framework to compare and contrast them. This paper serves an important purpose by connecting decades of "classical" research on consensus protocols to modern systems developed with Blockchain in mind. Our approach differs in scope and perspective, as we clarify not how Blockchain works technically but how analysts and users perceive it to work.

Another survey by Conti et al. focuses on security and privacy issues in Bitcoin\cite{Conti17}. The authors survey and tabulate known attacks against the Bitcoin network, miners, and clients as well as countermeasures and upgrade proposals that could improve privacy and anonymity in the system.  

Finally, Narayanan and Clark separated out the key technical ideas expressed in the design of Bitcoin and surveyed the academic lineage of each, repudiating the common belief that Bitcoin was a radical departure from existing research\cite{Narayanan17}. The authors discuss some aspects of how Bitcoin and Blockchain are understood by the general public, in particular identifying some key misconceptions that exist, but unlike in our work this is not their primary focus. Further, their survey covers academic literature while our goal of understanding how Blockchain is used and thought about by its users was best achieved by surveying analysis by industry and practitioners.