% !TEX root = ../main.tex

%------------------------Custom Commands----------------------%

% = = = Latin Short-forms (ie, eg, etc, et al)
\usepackage{xspace}
\newcommand{\etal}{\textit{et al.}\xspace}
\newcommand{\etc}{\textit{etc.}\xspace}
\newcommand{\ie}{\textit{i.e.,}\xspace}
\newcommand{\eg}{\textit{e.g.,}\xspace}
\newcommand{\cf}{\textit{cf.}\xspace}
\newcommand{\supra}{\textit{Supra}\xspace}

% = = = Arrow -> (\lt)
\newcommand{\lt}{$\rightarrow$\xspace}

% = = = Keywords (kw)
\newcommand{\kw}[1]{\textsf{#1}}

% = = = Colored text (textblue)
\newcommand{\textblue}[1]{\textcolor{blue}{#1}}

% = = = Compact Lists (compactlist, compactlistn)
\newenvironment{compactlist}
  {\begin{itemize} 
  \setlength{\itemsep}{0pt} 
  \setlength{\parskip}{0pt}} 
  {\end{itemize}}
  
\newenvironment{compactlistn}
  {\begin{enumerate} 
  \setlength{\itemsep}{0pt} 
  \setlength{\parskip}{0pt}} 
  {\end{enumerate}}
  
\renewcommand{\labelitemi}{$\bullet$}
  

%------------------------Crypto----------------------%  

% = = = Zp, Gq and Zq
\newcommand{\Zp}{\mathbb{Z}^{*}_{p}}
\newcommand{\Zq}{\mathbb{Z}_{q}}
\newcommand{\Gq}{\mathbb{G}_{q}}

% = = = Encryption, etc.
\newcommand{\Enc}[1]{\mathsf{Enc}(#1)}
\newcommand{\EncB}[1]{\llbracket #1 \rrbracket}
\newcommand{\ReRand}[1]{\mathsf{ReRand}(#1)}
\newcommand{\Hash}[1]{\mathcal{H}(#1)}
\newcommand{\Sign}[1]{\mathsf{Sig}(#1)}
\newcommand{\Comm}[1]{\mathsf{Comm}(#1)}
\newcommand{\Open}[1]{\mathsf{Open}(#1)}

% = = = Tuples
\newcommand{\tuple}[1]{\left \langle #1 \right \rangle}


%-------------------Custom for Paper----------------------%

% = = = Name
\newcommand{\Name}{\textsf{System Name}\xspace}

% = = = Concept Categories
\newcommand{\usecase}[1]{\textcolor{CornflowerBlue}{\textbf{#1}}} %try ProcessBlue, Cerulean, Cyan, or OliveGreen
\newcommand{\capability}[1]{\textcolor{Dandelion}{\textbf{#1}}}
\newcommand{\normproperty}[1]{\textcolor{Orange}{\textbf{#1}}}
\newcommand{\techproperty}[1]{\textcolor{NavyBlue}{\textbf{#1}}}
\newcommand{\primitive}[1]{\textcolor{Green}{\textbf{#1}}}


%-----------------------Notes------------------------------------%
\usepackage{todonotes}
\newcommand{\anote}[1]{\todo[inline]{Arkady:  #1}}
